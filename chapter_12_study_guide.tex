\documentclass[12pt]{article}
\pagestyle{empty}

\usepackage{graphicx}

\setlength\oddsidemargin{-0.5in}
\setlength\evensidemargin{-0.5in}
\setlength\textwidth{7in}
\setlength\textheight{10in}
\setlength\topmargin{0in}
\setlength\headheight{0in}
\setlength\headsep{0in}

\newcommand{\up}{$~$\vspace*{-0.7in}}
\newcommand{\ans}{\noindent Ans.\underline{\hspace{3in}}}
\newcommand{\liner}{\noindent\underline{\hspace*{7in}}}
\newcommand{\spac}{\hspace*{3em}}
\newcommand{\ds}{\displaystyle}
\newcommand{\halfspac}{\hspace*{1.5em}}

\begin{document}

\up

{\bf Ch 12 Study Guide  \hspace*{1em} Your Name: \_\_\_\_\_\_\_\_\_\_\_\_\_\_\_\_\_\_\_\_\_\_\_\_\_\_ \hspace*{1em} Class: 9am / 1pm}

\vspace*{0.2in}

\centerline{ \bf Calculus III - MATH 2630 - Spring 2013 \spac Instructor: Steven Clontz}

\vspace*{0.2in}

{\bf Draw a \framebox{box} around your answer. Show your work. Pens/calculators not allowed.}

\indent\liner


\begin{enumerate}

% (12.1) Open/closed domains
\item Give the domain and range of the function $f(x,y) = \frac{3}{x-y}$ in set notation or interval notation.

  \begin{itemize}
    \item Write the domain in set notation (3 pts)
    \item Write the range in set notation or interval notation (1 pt)
  \end{itemize}

\vspace*{2.5in}

Now, sketch the function's domain, tell if it is open/closed/both/neither, and tell if it is bounded/unbounded.

  \begin{itemize}
    \item Sketch the domain correctly (2 pts)
    \item Identify the domain as open/closed/both/neither (2 pts)
    \item Identify the domain as bounded/unbounded (2 pts)
  \end{itemize}

\vspace*{3in}

\liner

\newpage

% (12.1) Sketch Level Curves

\item Sketch three typical level curves for the function $f(x,y) = \frac{3}{x-y}$.

  \begin{itemize}
    \item Use the format $f(x,y)=c$ for level curves (2 pts)
    \item Simplify $f(x,y)=c$ into an identifiable curve equation (3 pts)
    \item Correctly express $f(x,y)=c$ for three values of $c$ (2 pts)
    \item Sketch each $f(x,y)=c$ in the $xy$-plane accurately (3 pts)
  \end{itemize}

\vspace*{7.6in}

\liner

\newpage

% (12.2) Finding Multidimensional Limits

\item Compute $\ds \lim_{(x,y)\to(0,0)} \frac{x\sin 2y - \sin 2y}{y-xy}$ without restricting to a path of approach.

  \begin{itemize}
    \item Correctly identify method to simplify fraction (4 pts)
    \item Execute method of simplification correctly (4 pts)
    \item Compute the correct value of the limit (2 pts)
  \end{itemize}

\vspace*{7in}

\liner

\newpage

% (12.2) Show Multidimensional limit DNE

\item Prove that $\ds \lim_{(x,y)\to(0,0)} \frac{3x^2y}{y^2+x^4}$ doesn't exist.

  \begin{itemize}
    \item Restrict the limit to two paths of approach, or a path of approach depending on $k$ \\ (3 pts)
    \item Conclude the limit DNE since the limit equals different real numbers along two different paths of approach (3 pts)
    \item Correctly compute the limits along two paths of approach (4 pts)
  \end{itemize}

\vspace*{7in}

\liner

\newpage

% (12.3) Finding Partial Derivatives

\item Find all of the first-order and second-order partial derivatives for $f(x,y) = x^3+3xy^2+e^y$.

  \begin{itemize}
    \item Compute each of $f_x$, $f_y$, $f_{xx}$, $f_{xy}=f_{yx}$, and $f_{yy}$ correctly (2 pts each)
  \end{itemize}

\vspace*{8in}

\liner

\newpage

% (12.4) Chain Rule

\item Find $\frac{df}{dt}$ at $t=\frac{\pi}{4}$ given $f(x,y,z)=\ln(xyz)$, $x=\cos t$, $y=\sec t$, and $z=4t^2$.

  \begin{itemize}
    \item If using Chain Rule:
      \begin{itemize}
        \item Use a correct Chain Rule formula $\frac{df}{dt}=\nabla f \cdot \frac{d\vec{r}}{dt}$ (2 pts)
        \item Compute each of $\frac{\partial f}{\partial x}$, $\frac{\partial f}{\partial y}$, $\frac{\partial f}{\partial z}$, $\frac{d x}{d t}$, $\frac{d y}{d t}$, $\frac{d z}{d t}$ correctly (1 pt each)
        \item Plug in value of $t$ and simplify (2 pts)
      \end{itemize}
    \item If using substitution:
      \begin{itemize}
        \item Plug $x,y,z$ into $f(x,y,z)$ to get $f$ as a function of $t$ (3 pts)
        \item Differentiate correctly (5 pts)
        \item Plug in value of $t$ and simplify (2 pts)
      \end{itemize}
  \end{itemize}

\vspace*{6in}

\liner

\newpage

% (12.5) Directional Derivatives

\item Find the directional derivative of $f(x,y,z)=3xy^2+6yz^2$ at the point $(1,2,-1)$ in the direction of $\left<2,-1,2\right>$.

  \begin{itemize}
    \item Use the formula $\frac{df}{ds_{\vec{u}}} = \nabla f \cdot \vec{u}$ (3 pts)
    \item Compute the direction $\vec{u}$ (2 pts)
    \item Compute the gradient $\nabla f$ (2 pts)
    \item Compute the directional derivative (3 pts)
  \end{itemize}

\vspace*{7in}

\liner

\newpage

% (12.6) Tangent Planes

\item Find the plane tangent to the surface $xy+2yz-xz=24$ at the point $(1,2,-1)$.

  \begin{itemize}
    \item Compute the gradient $\nabla f$ (3 pts)
    \item Use the plane equation $A(x-x_0)+B(y-y_0)+C(z-z_0)=0$ (4 pts)
    \item Plug into the plane equation correctly (3 pts)
  \end{itemize}

\vspace*{7in}

\liner

\newpage

% (12.7) Local Extrema and Saddle Points

\item Find and label all the points yielding local maximum values, local minimum values, and saddle points for $f(x,y)=x^3+3xy+y^3+2$.

  \begin{itemize}
    \item Compute the gradient $\nabla f$ (2 pts)
    \item Set $\nabla f = \vec{0}$ to find its critical points (1 pt)
    \item Find all the critical points (2 pts)
    \item Compute $f_{xx}, f_{yy}, f_{xy}$ correctly (1 pt)
    \item Compute $f_D$ correctly (1 pt)
    \item Use the 2nd derivative test to correctly label each critical point (1 pt for one, 3 pts for all)
  \end{itemize}

\vspace*{7in}

\liner

\newpage

% (12.8) Lagrange Multipliers

\item Find the radius and height of a cylinder with volume $2\pi$ such that the surface area is as small as possible. (HINT: Use the Method of Lagrange Multipliers along with the formulas $V=\pi r^2h$ and $SA=2\pi rh + 2\pi r^2$.)

  \begin{itemize}
    \item Identify and label the function $f$ to optimize (2 pt)
    \item Identify and label the restriction function $g=c$ (2 pt)
    \item Use the method of Lagrange Multipliers $\nabla f = \lambda \nabla g$ (3 pts)
    \item Find the optimizing input (3 pts)
  \end{itemize}

\vspace*{7in}

\liner

\newpage

%% Two pages of scratch work! %%
\centerline{Include extra scratch work below:}
\liner
\newpage
\centerline{Include extra scratch work below:}
\liner

\end{enumerate}

\end{document}
