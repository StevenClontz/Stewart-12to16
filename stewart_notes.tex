\newcommand{\thetitle}{
  Stewart's Calculus Chapter 12-16 | Lecture Notes
}

\documentclass[12pt]{article}

\usepackage{fancyhdr}

\usepackage{graphicx}

\pdfpagewidth 8.5in
\pdfpageheight 11in

\setlength\topmargin{0in}
\setlength\headheight{0in}
\setlength\headsep{0.2in}
\setlength\textheight{8in}
\setlength\textwidth{6in}
\setlength\oddsidemargin{0in}
\setlength\evensidemargin{0in}
\setlength\parindent{0in}
\setlength\parskip{0.1in} 

\pagestyle{fancy}
\headheight 35pt

\lhead{}
\chead{\thetitle}
\rhead{Page \thepage}

\lfoot{\footnotesize http://github.com/StevenClontz/Stewart-12to16}
\cfoot{}
\rfoot{\footnotesize Last updated on \today}
 
\usepackage{amssymb}
\usepackage{amsfonts}
\usepackage{amsmath}
\usepackage{mathtools}
\usepackage{amsthm}
\usepackage{wasysym} % for \smiley
      
\newcommand{\ds}{\displaystyle}
\newcommand{\vect}[1]{\mathbf{#1}}
\newcommand{\veci}{\mathbf{i}}
\newcommand{\vecj}{\mathbf{j}}
\newcommand{\veck}{\mathbf{k}}
\newcommand{\dvar}[1]{\,d{#1}}
\renewcommand{\d}[1]{\dvar{#1}}
\renewcommand{\div}{\textrm{div}\,}
\newcommand{\spin}{\textrm{spin}\,}
\newcommand{\curl}{\textrm{curl}\,}
\newcommand{\proj}{\textrm{proj}}
\newcommand{\<}{\left<}
\renewcommand{\>}{\right>}

\newcommand{\hruler}{\hrule\vspace{1em}}

\newcommand{\p}{\partial}
\newcommand{\pd}[2]{\frac{\p #1}{\p #2}}

\newcommand{\Arctan}{\text{Arctan }}
\newcommand{\Arcsin}{\text{Arcsin }}
\newcommand{\Arccos}{\text{Arccos }}
\newcommand{\Arcsec}{\text{Arcsec }}
\newcommand{\Arccsc}{\text{Arccsc }}
\newcommand{\Arccot}{\text{Arccot }}

\begin{document}

These notes were written to outline the major topics covered in Auburn University's Calculus III course based on Stewart's 7th Edition Calculus text. It progresses through most of the sections in Chapters 12 through 16, but Chapter 15 is reorganized slightly to introduce the Jacobian before introducing alternate coordinate systems. In addition, sections 15.5 and 15.6 are omitted entirely to match Auburn's course syllabus.

The purpose of these notes is not to replace any calculus or analysis textbook, but rather to be used as a guide/outline for students and instructors covering the topics in a Calculus III course.

As such, when deemed necessary, mathematical rigor is abandoned for the sake of simplicity or brevity. (Many theorems actually only apply to ``nice'' functions, usually requiring some level of continuity or differentiability.) Since for many applications of interest the relevant functions are ``nice'', students should be able to use these notes as a ``good-enough'' resource for working on computational problems, particularly the accompanying study problems.

\newpage

\centerline{\bf 12.1 Three-Dimensional Coordinate Systems}

  \begin{itemize}
  \item Distance between points in 3D space
    \[D = \sqrt{(x_2 - x_1)^2 + (y_2 - y_1)^2 + (z_2 - z_1)^2}\]
    
  \item Simple planes in 3D Space
    \[x=a,\, y=b,\, z=c\]
  
  \item Spheres in 3D Space
    \[(x-x_0)^2 + (y-y_0)^2 + (z-z_0)^2 = a^2\]
  \end{itemize}

\hr

\centerline{\bf 12.2 Vectors}

  \begin{itemize}
  \item Definition of a Vector
  
    \begin{itemize}
    \item A vector $\vect{v}=\overrightarrow{v}$ is a mathematical object which stores length (magnitude) and direction, and can be thought of as a directed line segment.
  
    \item Two vectors with the same length and direction are considered equal, even if they aren't in the same position. 

    \item We often assume the initial point lays at the origin.
    \end{itemize}
    
  \item Component Form
  
    The vector with initial point at $(0,0,0)$ and terminal point at $(v_x,v_y,v_z)$ is represented by \[\<v_x,v_y,v_z\>\]
    
  \item 2D and 3D Vectors
  
    \[\<a,b\>=\<a,b,0\>\]
    
  \item Position Vector
  
    If $P=(a,b,c)$ is a point, then $\vect{P}=\<a,b,c\>$ is its \textbf{position vector}. 

    We assume $(a,b,c)=\<a,b,c\>$.

  \item Vector Between Points
  
  The vector from $P_1 = (x_1,y_1,z_1)$ to $P_2 = (x_2,y_2,z_2)$ is \[\vect{P_1P_2} = \overrightarrow{P_1P_2} = \<x_2-x_1,y_2-y_1,z_2-z_1\>\]

  \newpage
  
  \item Length of a Vector
  
    \[|\vect{v}| = |\<v_1,v_2,v_3\>| = \sqrt{v_1^2 + v_2^2 + v_3^2}\]
  
  \item The Zero Vector
  
  \[\vect{0} = \overrightarrow{0} = \<0,0,0\>\]
  
  \item Vector Operations
    \begin{itemize}
    \item Addition
      \[\<v_1,v_2,v_3\> + \<u_1,u_2,u_3\> = \<v_1+u_1,v_2+u_2,v_3+u_3\>\]
    \item Scalar Multiplication
      \[k\<v_1,v_2,v_3\> = \<kv_1,kv_2,kv_3\> \]
    \end{itemize}

  \item Vector Operation Properties
    \begin{enumerate}
    \item $\vect{u}+\vect{v} = \vect{v}+\vect{u}$
    \item $(\vect{u}+\vect{v})+\vect{w} = \vect{u}+(\vect{v}+\vect{w})$
    \item $\vect{u}+\vect{0} = \vect{u}$
    \item $\vect{u}+(-\vect{u}) = \vect{0}$
    \item $0\vect{u} = \vect{0}$
    \item $1\vect{u} = \vect{u}$
    \item $a(b\vect{u}) = (ab)\vect{u}$
    \item $a(\vect{u} + \vect{v}) = a\vect{u} + a\vect{v}$
    \item $(a+b)\vect{u} = a\vect{u} + b\vect{u}$
    \end{enumerate}

  \item Unit Vectors
    \begin{itemize}
    \item A \textbf{unit vector} or \textbf{direction} is any vector whose length is $1$.
    
    \item Standard unit vectors
      \begin{itemize}
      \item $\veci = \<1,0,0\>$
      \item $\vecj = \<0,1,0\>$
      \item $\veck = \<0,0,1\>$
      \end{itemize}

    \item Standard Unit Vector Form:
      \[\<v_x,v_y,v_z\> = v_x\veci + v_y\vecj + v_z\veck\]

    \item Length-Direction Form:
      \[\vect{v} = |\vect{v}|\frac{\vect{v}}{|\vect{v}|}\]

    \end{itemize}
  \end{itemize}

\hr
  
\centerline{\bf 12.3 The Dot Product}
  
    \begin{itemize}
    \item Dot Product
      \[ \vect{u} \cdot \vect{v} = \<u_1,u_2,u_3\>\cdot\<v_1,v_2,v_3\> = u_1v_1 + u_2v_2 + u_3v_3 \]
  
    \item Angle between vectors
      \[\cos\theta = \frac{\vect{u}\cdot\vect{v}}{|\vect{u}||\vect{v}|}\] 
      
    \item Alternate Dot Product formula
      \[\vect{u} \cdot \vect{v} = |\vect{u}||\vect{v}|\cos \theta \]
      
    \item Orthogonal Vectors
      \begin{itemize}
      \item  $\vect{u},\vect{v}$ are orthogonal if $\vect{u} \cdot \vect{v} = 0$
      \item $\vect{u},\vect{v}$ are orthogonal if the angle between them is $\frac{\pi}{2} = 90^\circ$
      \item $\vect{0}$ is orthogonal to every vector
      \end{itemize}
    
    \item Dot Product Properties
      \begin{enumerate}
      \item $\vect{u} \cdot \vect{v} = \vect{v}\cdot\vect{u}$
      \item $(c\vect{u})\cdot \vect{v} = \vect{u} \cdot (c\vect{v}) = c(\vect{u} \cdot \vect{v})$
      \item $\vect{u} \cdot (\vect{v} + \vect{w}) = \vect{u}\cdot\vect{v} + \vect{u}\cdot \vect{w}$
      \item $\vect{u} \cdot \vect{u} = |\vect{u}|^2$
      \item $\vect{0} \cdot \vect{u} = 0$
      \end{enumerate}
    
    \item Projection Vector
      \[\proj_{\vect{v}}(\vect{u}) = \left(\frac{\vect{u}\cdot\vect{v}}{|\vect{v}|} \right)\frac{\vect{v}}{|\vect{v}|}\]

    \item Work
      \[W = \vect{F} \cdot \vect{D} = |\vect{F}||\vect{D}|\cos \theta\]\
    
    % All these are outdated from Hass - might replace or make new homework sheet.
    % \item \textbf{Suggested Exercises for 10.3}
    %   \begin{itemize}
    %   \item Finding and applying dot products: 1-8
    %   \item Work done by a constant vector force: 39-40
    %   \end{itemize}
    \end{itemize}
  
\newpage

\centerline{\bf 12.4 The Cross Product}
  
    \begin{itemize}
    \item Determinants
      \begin{itemize}
      \item 2x2 Determinant
      
      \[
\begin{array}{|c c|}
a & b \\
c & d \\
\end{array}
      = ad - bc
      \]
      
      \item 3x3 Determinant

\begin{center}\begin{tabular}{rl}
  $
    \begin{array}{|c c c|}
    a_1 & a_2 & a_3 \\
    b_1 & b_2 & b_3 \\
    c_1 & c_2 & c_3 \\
    \end{array}
  $
  &
  $
      = a_1 \,
    \begin{array}{|c c|}
    b_2 & b_3 \\
    c_2 & c_3 \\
    \end{array}
      - a_2 \,
    \begin{array}{|c c|}
    b_1 & b_3 \\
    c_1 & c_3 \\
    \end{array}
      + a_3 \,
    \begin{array}{|c c|}
    b_1 & b_2 \\
    c_1 & c_2 \\
    \end{array}
  $
  \\ [1ex] \\ &
  $
     = a_1 \,
    \begin{array}{|c c|}
    b_2 & b_3 \\
    c_2 & c_3 \\
    \end{array}
      + a_2 \,
    \begin{array}{|c c|}
    b_3 & b_1 \\
    c_3 & c_1 \\
    \end{array}
      + a_3 \,
    \begin{array}{|c c|}
    b_1 & b_2 \\
    c_1 & c_2 \\
    \end{array}
  $
  \\ [1ex] \\ &
  $= (a_1b_2c_3 + a_2b_3c_1 + a_3b_1c_2) - (a_3b_2c_1 + a_1b_3c_2 + a_2b_1c_3)$
\end{tabular}\end{center}
      
      \end{itemize}
    \item Cross Product
\[
    \vect{u} \times \vect{v} = 
\begin{array}{|c c c|}
\veci & \vecj & \veck \\
u_1 & u_2 & u_3 \\
v_1 & v_2 & v_3 \\
\end{array}
    =
    \<
\begin{array}{|c c|}
u_2 & u_3 \\
v_2 & v_3 \\
\end{array}
    \,,\,
\begin{array}{|c c|}
u_3 & u_1 \\
v_3 & v_1 \\
\end{array}
    \,,\,
\begin{array}{|c c|}
u_1 & u_2 \\
v_1 & v_2 \\
\end{array}
    \>
\]
    \[
    =
    \<u_2v_3-u_3v_2\,,\,u_3v_1-u_1v_3\,,\,u_1v_2-u_2v_1\>
    \]
    
    Shortcut ``long multiplication'' method:
    
    \[
\begin{array}{rcccccl}
\langle& u_1 & , & u_2 & , & u_3 & \rangle \\
\times\langle & v_1 & , & v_2 & , & v_3 & \rangle \\\hline
\langle & u_2v_3-u_3v_2 & , & u_3v_1-u_1v_3 & , & u_1v_2-u_2v_1 & \rangle
\end{array}
    \]

    \item Right-Hand Rule
      \begin{itemize}
      \item A method for determining a special orthogonal direction used throughout mathematics and physics in 3D space, with respect to an ordered pair of vectors $\vect{u},\vect{v}$
      \item $\vect{u}\times\vect{v}$ is orthogonal to both $\vect{u}$, $\vect{v}$ according to the Right-Hand Rule.
      \end{itemize}

    \newpage

    \item Cross Product Magnitude
      \[|\vect{u}\times\vect{v}|=|\vect{u}||\vect{v}|\sin\theta\]

      The area of the parallelogram determined by $\vect{u},\vect{v}$ is $|\vect{u}\times\vect{v}|$.
    
    \item Parallel Vectors
      \begin{itemize}
      \item  $\vect{u},\vect{v}$ are parallel if $\vect{u} \times \vect{v} = 0$
      \item $\vect{u},\vect{v}$ are parallel if the angle between them is $0=0^\circ$ or $\pi = 180^\circ$
      \item $\vect{0}$ is parallel to every vector
      \end{itemize}
    
    \item Cross Product Properties
    
      \begin{enumerate}
      \item $(r\vect{u}) \times (s\vect{v}) = (rs)(\vect{u} \times \vect{v})$
      \item $\vect{u} \times (\vect{v} + \vect{w}) = \vect{u} \times \vect{v} + \vect{u} \times \vect{w}$
      \item $(\vect{v} + \vect{w}) \times \vect{u} = \vect{v} \times \vect{u} + \vect{w} \times \vect{u}$
      \item $\vect{v} \times \vect{u} = -(\vect{u} \times \vect{v})$
      \item $\vect{0} \times \vect{u} = \vect{0}$
      \item $\vect{u} \times \vect{u} = \vect{0}$
      \end{enumerate}

    \item Standard Unit Vector Cross Products
      \begin{enumerate}
      \item $\veci \times \vecj = \veck$
      \item $\vecj \times \veck = \veci$
      \item $\veck \times \veci = \vecj$
      \end{enumerate}
      The standard unit vectors are known as a ``right handed frame''.
    
    \item Torque
    
    \[\overrightarrow{\tau} = \vect{r} \times \vect{F} \]
    \[|\overrightarrow{\tau}| = |\vect{r}||\vect{F}|\sin\theta\]
    
    \item Triple Scalar (or ``Box'') Product
    
    \[
    (\vect{u}\times\vect{v})\cdot\vect{w} =
    \begin{array}{|c c c|}
    u_1 & u_2 & u_3 \\
    v_1 & v_2 & v_3 \\
    w_1 & w_2 & w_3 \\
    \end{array}
    \]
    
    Its absolute value $|(\vect{u}\times\vect{v})\cdot\vect{w}|$ gives the volume of a parallelpiped determined by the three vectors.
    
    % \item \textbf{Suggested Exercises for 10.4}
    %   \begin{itemize}
    %   \item Finding cross products: 1-14
    %   \item Finding areas and unit normal vectors using cross products: 15-18
    %   \item Finding volumes using cross products: 19-22
    %   \item Computing torque: 25-26
    %   \end{itemize}
    \end{itemize}

\newpage

\centerline{\bf 12.5 Equations of Lines and Planes}
  
    \begin{itemize}
    \item Vector Equation and Parametric Equations for a Line
      \[\vect{r}(t) = \vect{P_0} + t\vect{v}\]
      \[x = x_0 + At, y = y_0 + Bt, z = z_0 + Ct\] for $-\infty < t < \infty$

    \item Symmetric Equations for a Line
      \[\frac{x-x_0}{A}=\frac{y-y_0}{B}=\frac{z-z_0}{C}\]
    
    \item Line Segment joining a pair of points
      \[\vect{r}(t) = \vect{P_0} + t(\vect{P_1}-\vect{P_0})=(1-t)\vect{P_0}+t\vect{P_1}\] for $0 \leq t \leq 1$
    
    \item Distance from a Point to a Line
      \[d = \frac{|\vect{PS} \times \vect{v}|}{|\vect{v}|}\]
    
    \item Equation for a Plane
      \[A(x-x_0) + B(y-y_0) + C(z-z_0) = 0\]
      \[Ax+By+Cz = D\]
    
    \item Line of Intersection of Two Planes
      \[\vect{r}(t) = \vect{P_0} + t(\vect{n_1} \times \vect{n_2})\]

    \item Angle of Intersection of Two Planes
      \[\cos\theta = \frac{\vect{n_1}\cdot\vect{n_2}}{|\vect{n_1}||\vect{n_2}|}\]
    
    \item Distance from a Point to a Plane
      \[d = \frac{|\vect{PS} \cdot \vect{n}|}{|\vect{n}|}\]

    % \item \textbf{Suggested Exercises for 10.5}
    %   \begin{itemize}
    %   \item Finding parametric equations for lines: 1-12
    %   \item Finding parametrizations for line segments: 13-20
    %   \item Finding equations for planes: 21-26
    %   \item Distance from a point to a line: 33-38
    %   \item Distance from a point to a plane: 39-44
    %   \end{itemize}
    
    \end{itemize}


\newpage

\centerline{\bf 12.6 Cylinders and Quadratic Surfaces }

\begin{itemize}
\item Sketching surfaces

  \begin{itemize}
  \item To sketch a 3D surface, sketch planar cross-sections
    \begin{itemize}
    \item $z=c$ is parallel to $xy$ plane
    \item $y=b$ is parallel to $xz$ plane
    \item $x=a$ is parallel to $yz$ plane
    \end{itemize}
  \end{itemize}

\item Cylinders

  \begin{itemize}
  \item A \textbf{cylinder} is any surface generated by considering parallel lines passing through a planar curve.
  \item A 3D surface defined by a function of only two variables results in a cylinder.
  \end{itemize}

\item Quadric Surfaces

  \begin{itemize}
  \item A \textbf{quadric surface} is any surface defined by a second degree equation of $x,y,z$.
  \item Most helpful to consider the cross-sections in each of the coordinate planes.
  \end{itemize}

\item Ellipsoids
  \begin{itemize}
  \item Cross-sections in the coordinate planes include
    \begin{itemize}
    \item Three ellipses
    \end{itemize}
  \end{itemize}

\item Elliptical Cone
  \begin{itemize}
  \item Cross-sections in the coordinate planes include
    \begin{itemize}
    \item Two double-lines
    \item One point (with parallel ellipses)
    \end{itemize}
  \end{itemize}

\item Elliptical Paraboloid
  \begin{itemize}
  \item Cross-sections in the coordinate planes include
    \begin{itemize}
    \item Two parabolas
    \item One point (with parallel ellipses)
    \end{itemize}
  \end{itemize}
\newpage
\item Hyperbolic Paraboloid
  \begin{itemize}
  \item Cross-sections in the coordinate planes include
    \begin{itemize}
    \item Two parabolas (with parallel parabolas)
    \item One double line (with parallel hyperbolas)
    \end{itemize}
  \end{itemize}

\item Hyperboloid of One Sheet
  \begin{itemize}
  \item Cross-sections in the coordinate planes include
    \begin{itemize}
    \item Two hyperbolas
    \item One ellipsis (with parallel ellipses)
    \end{itemize}
  \end{itemize}

\item Hyperboloid of Two Sheets
  \begin{itemize}
  \item Cross-sections in the coordinate planes include
    \begin{itemize}
    \item Two hyperbola
    \item One empty cross-section (with parallel ellipses)
    \end{itemize}
  \end{itemize}

% \item \textbf{Suggested Exercises for 10.6}
%   \begin{itemize}
%   \item Identify surfaces from equations: 1-12
%   \item Sketching surfaces: 13-44
%   \end{itemize}
\end{itemize}

\newpage

\centerline{\bf 13.1 Vector Functions and Space Curves}

\begin{itemize}
  \item Curves, Paths, and Vector Functions
    \begin{itemize}
      \item A \textbf{position function} maps a moment in time to a position on a path. It can be defined with \textbf{parametric equations} \[x=x(t), y=y(t), z=z(t)\] or with a \textbf{vector function} \[\vect{r}(t) = \<x(t),y(t),z(t)\>\]
      \item $x(t),y(t),z(t)$ are called \textbf{component functions}
    \end{itemize}

  \item Vector Function Limits
    \[\lim_{t\to a} \vect{r}(t) = \<\lim_{t\to a} f(t), \lim_{t\to a} g(t), \lim_{t\to a} h(t)\>\]

  \item Continuity of Vector Functions
    \begin{itemize}
      \item
      The function $\vect{r}(t)$ is \textbf{continuous} if \[\lim_{t\to a}\vect{r}(t) = \vect{r}(a)\] for all $a$ in its domain.
      \item $\vect{r}(t)$ is continuous exactly when $f(t),g(t),h(t)$ are all continuous.
    \end{itemize}
\end{itemize}

\newpage

\centerline{\bf 13.2 Derivatives and Integrals of Vector Functions}

\begin{itemize}
  \item Derivatives of Vector Functions
      \[\ds \frac{d\vect{r}}{dt} = \vect{r}'(t) = \lim_{\Delta t \to 0} \frac{\vect{r}(t+\Delta t) - \vect{r}(t)}{\Delta t} = \<f'(t),g'(t),h'(t)\>\]
    \begin{itemize}
      \item $\vect{r}(t)$ is \textbf{differentiable} if $\vect{r}'(t)$ is defined for every value of $t$ is in its domain.
      \item $\vect{r}'(a)$ is a \textbf{tangent vector} to the curve where $t=a$
      \item The \textbf{tangent line} to a curve at $t=a$: \[\vect{l}(t)=\vect{r}(a)+t\vect{r}'(a)\]
    \end{itemize}

  % \item Vectors and Physics
  %   \begin{itemize}
  %   \item Position: $\vect{r}(t)$
  %   \item Velocity: $\vect{v}(t) = \vect{r}'(t) = \frac{d\vect{r}}{dt}$
  %   \item Speed: $|\vect{v}(t)|$ 
  %   \item Direction: $\frac{\vect{v}(t)}{|\vect{v}(t)|}$ 
  %     \begin{itemize}
  %       \item (Remember that $\vect{v}=|\vect{v}|\frac{\vect{v}}{|\vect{v}|}$)
  %     \end{itemize}
  %   \item Acceleration: $\vect{a}(t) = \vect{v}'(t) = \vect{r}''(t)$ 
  %   \end{itemize}
  
  \item Differentiation Rules for Vector Functions
      \[\frac{d}{dt} [\vect{C}] = \vect{0}\]
      \[\frac{d}{dt} [c\vect{u}(t)] = c\vect{u}'(t)\]
      \[\frac{d}{dt} [f(t)\vect{C}] = f'(t)\vect{C}\]
      \[\frac{d}{dt} [\vect{u}(t) \pm \vect{v}(t)] = \vect{u}'(t) \pm \vect{v}'(t)\]
      \[\frac{d}{dt} [f(t)\vect{u}(t)] = f(t)\vect{u}'(t) + f'(t)\vect{u}(t)\]
      \[\frac{d}{dt} [\vect{u}(t) \cdot \vect{v}(t)] = \vect{u}(t)\cdot\vect{v}'(t) + \vect{u}'(t)\cdot\vect{v}(t)\]
      \[\frac{d}{dt} [\vect{u}(t) \times \vect{v}(t)] = \vect{u}(t)\times\vect{v}'(t) + \vect{u}'(t)\times\vect{v}(t)\]
      \[\frac{d\vect{u}}{dt} = \frac{d}{dt} [\vect{u}(f(t))] =\vect{u}'(f(t))f'(t) = \frac{d\vect{u}}{df}\frac{df}{dt}\]
  
  \item Derivative of a Constant Length Vector Function
    \begin{itemize}
      \item If $|\vect{r}(t)|=c$ always, then \[\vect{r}(t) \cdot \vect{r}'(t) = 0\]
      \item Thus the derivative of a constant length vector function is perpindicular to the original.
    \end{itemize}
  
  % \item \textbf{Suggested Exercises for 11.1}
  %   \begin{itemize}
  %     \item Position/Velocity/Acceleration Vectors: 1-14
  %   \end{itemize}

    \item Antiderivatives of Vector Functions
      \begin{itemize}
        \item If $\vect{R}'(t)=\vect{r}(t)$, then $\vect{R}(t)$ is an \textbf{antiderivative} of $\vect{r}(t)$.
        \item The \textbf{indefinite integral} $\ds \int \vect{r}(t) \dvar{t}$ is the collection of all the antiderivatives of $\vect{r}(t)$.
        \[\ds\int \vect{r}(t) \dvar{t} = \vect{R}(t) + \vect{C}\]
        \[\ds\int \vect{r}(t) \dvar{t} = \<\int x(t) \dvar{t}, \int y(t) \dvar{t}, \int z(t) \dvar{t} \> \]
      \end{itemize}

    \item Definite Integrals
      \[\ds\int^b_a \vect{r}(t) \dvar{t} = \<\int^b_a x(t) \dvar{t}, \int^b_a y(t) \dvar{t}, \int^b_a z(t) \dvar{t} \> \]
      \[\int^b_a \vect{r}(t)dt = \left[\vect{R}(t)\right]^b_a=\vect{R}(b)-\vect{R}(a)\]

    \item Differential Vector Equations
      \begin{itemize}
      \item If we know $\vect{r}'(t)$ and $\vect{r}(a)$ for some $t=a$, then \[\vect{r}(t)=\int_a^t\vect{r}'(t)\,dt+\vect{r}(a)\]
      \end{itemize}
    
    % \item Ideal Projectile Motion 
    %   \begin{itemize}
    %   \item Assume the following:
    %     \begin{itemize}
    %       \item The acceleration acting on a projectile is $\<0,-g\>$
    %       \item The launch position is the origin $\<0,0\>=\vect{0}$
    %       \item The launch angle is $\alpha$
    %       \item The initial velocity is $\vect{v_0}$, and initial speed is $v_0=|\vect{v_0}|$
    %     \end{itemize}
    %   \item This results in the initial value problem:
    %     \[\vect{a}(t) = \<0,-g\>\]
    %     \[\vect{v}(0) = \<v_0\cos\alpha,v_0\sin\alpha\>\]
    %     \[\vect{r}(0) = \<0,0\> \]
    %   \item The velocity function solves to \[\vect{v}(t) = \<v_0\cos\alpha,-gt+v_0\sin\alpha\>\]
    %   \item The position function solves to 
    %     \[\vect{r}(t) = \<(v_0\cos\alpha)t,-\frac{1}{2}gt^2+(v_0\sin\alpha)t\>\]
    %     with parametric equations 
    %     \[x=(v_0\cos\alpha)t\] \[y=-\frac{1}{2}gt^2+(v_0\sin\alpha)t\]
    %   \item The parabolic position curve can be expressed as \[y = -\left(\frac{g}{2v_0^2\cos^2\alpha}\right)x^2+(\tan\alpha)x\]
    %   \item Properties of ideal projectile motion beginning at origin:
    %     \[y_{max} = \frac{(v_0\sin\alpha)^2}{2g}\]
    %     \[t_{tot} = \frac{2v_0\sin\alpha}{g}\]
    %     \[R = \frac{v_0^2}{g}\sin2\alpha\]
    %   \item If we assume the initial position is instead $\vect{r}(0)=\<x_0,y_0\>$, then the position function changes to \[\vect{r}(t)=\<(v_0\cos\alpha)t+x_0,-\frac{1}{2}gt^2+(v_0\sin\alpha)t+y_0\>\]
    %   \end{itemize}
    % \item \textbf{Suggested Exercises for 11.2}
    %   \begin{itemize}
    %   \item Vector function integrals: 1-6
    %   \item Vector function initial value problems: 7-12
    %   \item Ideal projectile motion: 15-21
    %   \end{itemize}
  \end{itemize}

\newpage

\centerline{\bf 13.3 Arc Length and Curvature}

  \begin{itemize}
    \item Arc Length along a Space Curve
          \[L = \int_a^b \left|\lim_{\Delta{t}\to0}\frac{\vect{r}(t+\Delta{t})-\vect{r}(t)}{\Delta{t}}\right| \dvar{t} = \int_a^b |\vect{r}'(t)| \dvar{t}\]

    \item Arclength Parameter 
      \[s(t) = \int_0^t |\vect{r}'(u)|du\]
      \[\frac{ds}{dt} = |\vect{r}'(t)|\]

    \item Unit Tangent Vector
      \[\vect{T}(s) = \frac{d\vect{r}}{ds}\]
      \[\vect{T}(t) = \frac{d\vect{r}/dt}{|d\vect{r}/dt|}\]

%     \item \textbf{Suggested Exercises for 11.3}
%       \begin{itemize}
%       \item Unit tangent vectors and arc length: 1-8
%       \item Arc length parameter: 11-14
%       \end{itemize}
%   \end{itemize}

% \newpage

% \centerline{\bf 11.4 Curvature of a Curve}

%   \begin{itemize}

    \item Curvature
      \[
        \kappa(s) = \left|\frac{d\vect{T}}{ds}\right| 
      \]
      \[
        \kappa(t) = \frac{|d\vect{T}/dt|}{|d\vect{r}/dt|} = \frac{|\frac{d\vect{r}}{dt}\times \frac{d^2\vect{r}}{dt^2}|}{|\frac{d\vect{r}}{dt}|^3}
      \]
      For $y=f(x)$:
      \[
        \kappa(x) = \frac{|f''(x)|}{[1+(f'(x))^2]^{3/2}}
      \]
    
    % \item Curvature of a Circle
    
    %   \begin{itemize}
    %   \item The curvature of a circle with radius $a$ is constantly \[\kappa = \frac{1}{a}\]
    %   \end{itemize}
    
    \item Principal Unit Normal Vector
      \[\vect{N}(s) = \frac{d\vect{T}/ds}{|d\vect{T}/ds|}\]
      \[\vect{N}(t) = \frac{d\vect{T}/dt}{|d\vect{T}/dt|}\]
    
    % \item Circles of Curvature
    
    %   \begin{itemize}
    %   \item The circle which:
    %     \begin{enumerate}
    %     \item is tangent to a curve at a point
    %     \item has the same curvature as the curve at that point
    %     \item lies on the concave side of the curve, in the direction of $\vect{N}$
    %     \end{enumerate}
    %   \item Radius: $a = \displaystyle\frac{1}{\kappa}$.
    %   \item Center: $\<x_0,y_0\> = \vect{r}(t_0)+a\vect{N}$.
    %   \item Equations: 
    %     \[(x-x_0)^2+(y-y_0)^2=a^2\]
    %     \[\vect{c}(t) = \<a\sin t+x_0,a\cos t+y_0\>, 0\leq t\leq 2\pi\]
    %   \end{itemize}
      
    % \item \textbf{ Suggested Exercises for 11.4:}
    
    %   \begin{itemize}
    %   \item Find $\vect{T},\vect{N},\kappa$: 1-4, 9-16
    %   \item Circles of Curvature: 21-22
    %   \end{itemize}
    
%   \end{itemize}

% \newpage

% \centerline{\bf 11.5 Tangental and Normal Components of Acceleration}

%   \begin{itemize}
  
    \item Binormal Unit Vector
      \[\vect{B} = \vect{T} \times \vect{N}\]
      The triple $\vect{T},\vect{N},\vect{B}$ forms a right-handed frame.

  \end{itemize}

\newpage

\centerline{\bf 13.4 Motion in Space: Velocity and Acceleration}

  \begin{itemize}

    \item Position, Velocity, and Acceleration
      \begin{itemize}
      \item Position: $\vect{r}(t)$
      \item Velocity: $\vect{v}(t) = \vect{r}'(t) = \frac{d\vect{r}}{dt}$
      \item Speed: $v(t)=|\vect{v}(t)|=\frac{ds}{dt}$ 
      \item Direction: $\vect{T}(t)=\frac{\vect{v}(t)}{|\vect{v}(t)|}$ 
      \item Acceleration: $\vect{a}(t) = \vect{v}'(t) = \vect{r}''(t)$ 
      \end{itemize}

    \item Ideal Projectile Motion
      \[\vect{a}(t)=\<0,-g\>\]
      \[\vect{v}(t)=\<v_0\cos\alpha,-gt+v_0\sin\alpha\>\]
      \[\vect{r}(t)=\<(v_0\cos\alpha)t,-\frac{1}{2}gt^2+(v_0\sin\alpha)t\>\]
      
    \item Tangental and Normal Components of Acceleration
      \[\vect{a} = \left(\frac{d^2s}{dt^2}\right)\vect{T} + \kappa\left(\frac{ds}{dt}\right)^2\vect{N}+0\vect{B}\]
      \begin{itemize}
        \item Tangental component
          \[a_T = \frac{d^2s}{dt^2} = v' \]
        \item Normal component
          \[a_N = \kappa\left(\frac{ds}{dt}\right)^2 = \kappa v^2 = \sqrt{|\vect{a}|^2 - a_T^2}\]
      \end{itemize}
    
    % \item Torsion
    %   \begin{itemize}
    %     \item Magnitude of torsion
    %       \[|\tau| = \left|\frac{d\vect{B}}{ds}\right|\]
    %     \item Signed torsion
    %       \[\frac{d\vect{B}}{ds} = (-\tau)\vect{N}\]
    %       \[\tau = -\frac{d\vect{B}}{ds}\cdot \vect{N} = -\frac{1}{|\vect{v}|}\left(\frac{d\vect{B}}{dt}\cdot\vect{N}\right)\]
    %       \[
    %         \tau
    %         =
    %         \frac{
    %         \begin{array}{|ccc|}
    %         \dot{x} & \dot{y} & \dot{z} \\
    %         \ddot{x} & \ddot{y} & \ddot{z} \\
    %         \dddot{x} & \dddot{y} & \dddot{z}
    %         \end{array}
    %         }{
    %         |\vect{v}\times\vect{a}|^2
    %         }
    %       \]
    %   \end{itemize}
      
    % \item \textbf{ Suggested Exercises for 11.5:}
    
    %   \begin{itemize}
    %   \item Finding tangental and normal components of acceleration: 1-6
    %   \item Finding $\vect{B}$ and $\tau$: 9-16
    %   \end{itemize}
  
  \end{itemize}

% \newpage

% \centerline{\bf 11.6 Velocity and Acceleration in Polar Coordinates}

%   \begin{itemize}
  
%     \item Polar Coordinates $(r,\theta)$
%       \begin{itemize}
%         \item Cartesian to Polar
%           \[r = x^2+y^2, \theta = \Arctan\left(\frac{y}{x}\right)\]
%         \item Polar to Cartesian
%           \[x = r\cos\theta, y=r\sin\theta\]
%       \end{itemize}
      
%     \item Cylindrical Coordinates $(r,\theta,z)$
%       \begin{itemize}
%         \item Cartesian to Cylindrical
%           \[r = x^2+y^2, \theta = \Arctan\left(\frac{y}{x}\right), z=z\]
%         \item Cylindrical to Cartesian
%           \[x = r\cos\theta, y=r\sin\theta, z=z\]
%       \end{itemize}
      
%     \item Polar/Cylindrical Unit Vectors
%       \[\vect{u}_r = \<\cos\theta,\sin\theta\>, \vect{u}_\theta = \<-\sin\theta,\cos\theta\>\]
%       \begin{itemize}
%         \item Cylindrical Right-handed frame
%           \[\vect{u}_r,\vect{u}_\theta,\veck\]
%         \item Derivatives
%           \[\frac{d}{dt}\left[\vect{u}_r\right] = \dot{\vect{u}_r} = \dot{\theta}\vect{u}_\theta \]
%           \[\frac{d}{dt}\left[\vect{u}_\theta\right] = \dot{\vect{u}_\theta} = -\dot{\theta}\vect{u}_r \]
%         \item Polar Position/Velocity/Acceleration
%           \[\vect{r} = r\vect{u}_r\]
%           \[\vect{v} = \dot{r}\vect{u}_r + r\dot\theta\vect{u}_\theta\]
%           \[\vect{a} = (\ddot{r} - r\dot\theta^2)\vect{u}_r + (r\ddot\theta + 2\dot{r}\dot\theta)\vect{u}_\theta\]
%         \item Cylindrical Position/Velocity/Acceleration
%           \[\vect{r} = r\vect{u}_r + z\veck\]
%           \[\vect{v} = \dot{r}\vect{u}_r + r\dot\theta\vect{u}_\theta + \dot{z}\veck\]
%           \[\vect{a} = (\ddot{r} - r\dot\theta^2)\vect{u}_r + (r\ddot\theta + 2\dot{r}\dot\theta)\vect{u}_\theta+\ddot{z}\veck\]
%       \end{itemize}
    
%     \item \textbf{ Suggested Exercises for 11.6:}
%       \begin{itemize}
%       \item Expressing $\vect{v}$ and $\vect{a}$ in terms of $\vect{u}_r$ and $\vect{u}_\theta$: 1-5
%       \end{itemize}
%   \end{itemize}



\newpage

\centerline{\bf 14.1 Functions of Several Variables}

\begin{itemize}

  \item Functions of Two Variables
    \begin{itemize}
      \item A \textbf{function $f$ of two variables} is a rule which assigns a real number $f(x,y)$ to each pair of real numbers $(x,y)$ in its \textbf{domain} \[\text{dom}(f)\subseteq \mathbb{R}^2\] The set of values $f$ takes on is its \textbf{range} \[\text{ran}(f) = \{f(x,y):(x,y)\in\text{dom}(f)\}\]
      \item The \textbf{level curve} for each $k\in\text{ran}(f)$ is given by the equation \[f(x,y)=k\]
      \item The \textbf{graph} of $f$ is a surface in 3D space which visualizes the function, given by the equation $z=f(x,y)$.
    \end{itemize}

  \item Functions of Three Variables
    \begin{itemize}
      \item A \textbf{function $f$ of three variables} is a rule which assigns a real number $f(x,y,z)$ to each pair of real numbers $(x,y,z)$ in its \textbf{domain} \[\text{dom}(f)\subseteq \mathbb{R}^3\] The set of values $f$ takes on is its \textbf{range} \[\text{ran}(f) = \{f(x,y,z):(x,y,z)\in\text{dom}(f)\}\]
      \item The \textbf{level surface} for each $k\in\text{ran}(f)$ is given by the equation \[f(x,y,z)=k\]
    \end{itemize}

  \item Alternate Forms
    \begin{itemize}
      \item We may also consider functions of the form $f(x_1,x_2,\dots)=f(P)=f(\vect{r})$.
      \item If $P=(x,y)$ and $\vect{r}=\<x,y\>$, then $f(x,y)=f(P)=f(\vect{r})$. 
      \item If $P=(x,y,z)$ and $\vect{r}=\<x,y,z\>$, then $f(x,y,z)=f(P)=f(\vect{r})$. 
    \end{itemize}



%   \item Real-Valued Functions

%     \begin{itemize}
%       \item A \textbf{real-valued function} $f$ on with \textbf{domain} $D \subset \mathbb{R}^n$ is a rule that assigns a real number \[f(x_1,x_2,\dots,x_n) \in \mathbb{R}\] to each $(x_1,x_2,\dots,x_n) \in D$.
%       \item The domain of a function is assumed to be all of $\mathbb{R}^n$ except where the function is not well-defined.
%       \item The \textbf{range} of the function is \[R = \{f(x_1,x_2,\dots,x_n) : (x_1,x_2,\dots,x_n) \in D\}\]
%     \end{itemize}
  
%   \item Regions
  
%     \begin{itemize}
%       \item A subset of the $xy$-plane ($\mathbb{R}^2$) or $xyz$-space ($\mathbb{R}^3$) is known as a \textbf{region}.
%       \item The \textbf{ball} $B(p,\epsilon)$ is the set of points \[B(p,\epsilon) = \{q \in \mathbb{R}^2 : \text{the distance between }p\text{ and }q\text{ is less than }\epsilon\}\] Its \textbf{center} is the point $p$ and its \textbf{radius} is $\epsilon$.
%       \item A point $p\in\mathbb{R}^2$ is known as an \textbf{interior point} of a region $R$ if \textit{there exists some ball} containing $p$ that lies inside $R$.
%       \item A point $p\in\mathbb{R}^2$ is known as a \textbf{boundary point} of a region $R$ if \textit{every ball} containing $p$ contains some points in $R$ and some points not in $R$.
%       \item A point $p\in\mathbb{R}^2$ is known as an \textbf{exterior point} of a region $R$ if \textit{there exists some ball} containing $p$ that lies outside $R$.
%       \item The \textbf{interior} of $R$ is the set \[\textrm{int}(R)=\{p : p \text{ is an interior point of } R\}\]
%       \item The \textbf{boundary} of $R$ is the set \[\textrm{bd}(R)=\{p : p \text{ is a boundary point of } R\}\]
%       \item The \textbf{exterior} of $R$ is the set \[\textrm{ext}(R)=\{p : p \text{ is an exterior point of } R\}\]
%       \item A region $R$ is \textbf{open} if it doesn't contain any of its boundary.
%       \item A region $R$ is \textbf{closed} if it contains all of its boundary.
%       \item A region $R$ is \textbf{bounded} if it can be contained within a ball.
%       \item A region $R$ is \textbf{unbounded} if it cannot be contained within a ball.
%     \end{itemize}

%   \item Sketching Functions

%     \begin{itemize}
%       \item Level curve
%         \[\{(x,y):f(x,y)=c\}\]
%       \item Surface $z=f(x,y)$
%         \[\{(x,y,f(x,y)): (x,y)\in \textrm{Dom}(f)\}\]
%       \item Contour curve 
%         \[\{(x,y,c): f(x,y)=c\}\]
%       \item Level surface
%         \[\{(x,y,z):f(x,y,z)=c\}\]
%     \end{itemize}

% \item \textbf{ Suggested Exercises for 12.1:}

%   \begin{itemize}
%   \item Identifying and describing domains, ranges, level curves, boundaries: 1-12
%   \item Relating level curves to graphs: 13-18
%   \item Sketching surfaces and level curves: 19-28
%   \item Finding level curves through a point: 29-32
%   \item Sketching level surfaces: 33-40
%   \item Finding level surfaces through a point: 41-44
%   \end{itemize}
  
\end{itemize}

\newpage

\centerline{\bf 14.2 Limits and Continuity}

\begin{itemize}

  \item Limits
    \begin{itemize}
      \item If the value of the function $f(P)$ becomes arbitrarily close to the number $L$ as vectors $P$ close to $P_0$ are plugged into the function, then the \textbf{limit of $f(P)$ as $P$ approaches $P_0$} is $L$: \[\lim_{P\to P_0} f(P) = L\]
      \item For functions of two or three variables:
      \[\lim_{(x,y)\to(x_0,y_0)}f(x,y) = L\]
      \[\lim_{(x,y,z)\to(x_0,y_0,z_0)}f(x,y,z) = L\]
    \end{itemize}
  
  \item Showing a Limit DNE
  
    \begin{itemize}
      \item In order for a limit $\lim_{P\to P_0}f(x,y)$ to exist, the values of $f$ must approach $L$ no matter which direction we approach $P_0$.
      \item Choose $y=g(x)$ and $y=h(x)$ where $P_0$ lays on both graphs. If
        \[
          \lim_{x\to x_0} f(x,g(x)) \not= \lim_{x\to x_0} f(x,h(x))
        \]
        then $\ds\lim_{P\to P_0} f(x,y)$ DNE.
      \item Or choose $x=g(y)$ and $x=h(y)$ where $P_0$ lays on both graphs. If 
        \[
          \lim_{y\to y_0} f(g(y),y) \not= \lim_{y\to y_0} f(h(y),y)
        \]
        then $\ds\lim_{P\to P_0} f(x,y)$ DNE.
    \end{itemize}
    
  \item Limit Laws
  
    \[\lim_{P\to P_0}(f(P)\pm g(P)) = \lim_{P\to P_0}f(P) \pm \lim_{P\to P_0}g(P)\]
    \[\lim_{P\to P_0}(f(P)\cdot g(P)) = \lim_{P\to P_0}f(P) \cdot \lim_{P\to P_0}g(P)\]
    \[\lim_{P\to P_0}(kf(P)) = k\lim_{P\to P_0}f(P)\]
    \[\lim_{P\to P_0}\frac{f(P)}{g(P)} = \frac{\ds \lim_{P\to P_0}f(P)}{\ds \lim_{P\to P_0}g(P)}\]
    \[\lim_{P\to P_0}(f(P))^{r/s} = \left(\lim_{P\to P_0}f(P)\right)^{r/s}\]
          
  \item Computing Limits
    
      \begin{itemize}
      \item Variables not involved in a limit may be eliminated: \[\lim_{P \to P_0} f(x) = \lim_{x\to x_0} f(x)\]
      \item Due to the Limit Laws, many limits follow the ``just plug it in'' rule.
      \item If plugging in results in a zero in a denominator, use factoring, perhaps with conjugates.
      \item L'Hopital's Rule does not apply for multiple variable limits.
      \end{itemize}
  
  \item Continuity
  
    \begin{itemize}
      \item A function $f(P)$ is \textbf{continuous} if $\ds \lim_{P\to P_0}f(P) = f(P_0)$ for all points $P_0$ in its domain.
      \item If a multi-variable function is composed of continuous single-variable functions, then it is also continuous.
    \end{itemize}
  
  % \item \textbf{ Suggested Exercises for 12.2:}
  
  %   \begin{itemize}
  %   \item Computing limits: 1-26
  %   \item Showing limits don't exist: 35-42
  %   \end{itemize}

\end{itemize}

\newpage

\centerline{\bf 14.3 Partial Derivatives}

\begin{itemize}

  \item Partial Derivatives
    \begin{itemize}
      \item For a function $f$ of two variables $(x,y)$:
        \[\frac{\p f}{\p x}=f_x(x,y)=\lim_{h\to0}\frac{f(x+h,y)-f(x,y)}{h}\]
        \[\frac{\p f}{\p y}=f_y(x,y)=\lim_{h\to0}\frac{f(x,y+h)-f(x,y)}{h}\]
      \item To compute partial derivatives with respect to a variable, treat all other variables as constants and differentiate as normal.
      \item Functions of more than two variables behave similarly. For $T(x,y,z)$:
        \[\frac{\p T}{\p z}=T_z(x,y,z)=\lim_{h\to0}\frac{T(x,y,z+h)-T(x,y,z)}{h}\]
    \end{itemize}
    
  \item Higher Order Partial Derivatives

    \[
      \frac{\p^2 f}{\p x\p y} = \frac{\p}{\p x}\left[ \frac{\p f}{\p y} \right] = (f_y)_x = f_{yx}
    \] 
    \[
      \frac{\p^2 g}{\p z^2}= \frac{\p}{\p z}\left[ \frac{\p g}{\p z} \right]= (g_z)_z=g_{zz}
    \]
  
    \item Mixed Derivative Theorem 
      \begin{itemize}
      \item For many naturally occuring functions:
        \[f_{xy}=f_{yx}\]
      \end{itemize}
  
  % \item \textbf{Suggested Exercises for 12.3:}
  
  %   \begin{itemize}
  %     \item Finding first-order partial derivatives: 1-38
  %     \item Finding second-order partial derivatives: 41-50
  %     \item Finding partial derivatives from the limit definition: 53-56
  %   \end{itemize}

\end{itemize}

\newpage

\centerline{\bf 14.4 Tangent Planes and Linear Approximations}

\begin{itemize}

  \item Tangent Plane to $z=f(x,y)$ at $(a,b,f(a,b))$
    \[
      z-f(a,b)=f_x(a,b)(x-a)+f_y(a,b)(y-b)
    \]

  \item Linearization of $f(x,y)$ at $(a,b)$
    \[
      L(x,y) = f(a,b)+f_x(a,b)(x-a)+f_y(a,b)(y-b)
    \]

  \item Differentiability and a Sufficient Condition
    \begin{itemize}
      \item A multi-variable function $f$ is \textbf{differentiable} at a point if its linearizaration approximates the value of the function near that point.
      \item If $f_x$, $f_y$ exist near $(a,b)$ and are continuous at $(a,b)$, then $f$ is differentiable at $(a,b)$.
    \end{itemize}

  \item Linear Approximation

    If $f$ is differentiable at $(a,b)$, then
      \[
        f(x,y) \approx f(a,b)+f_x(a,b)(x-a)+f_y(a,b)(y-b)
      \]

\end{itemize}

\newpage

\centerline{\bf 14.5 The Chain Rule}

\begin{itemize}

  \item Gradient Vector Function
    \[\nabla f(x,y) = \< f_x(x,y),f_y(x,y) \> = \<\frac{\p f}{\p x}, \frac{\p f}{\p y}\>\]
    \[\nabla f(x,y,z) = \< f_x(x,y,z),f_y(x,y,z),f_z(x,y,z) \> = \<\frac{\p f}{\p x}, \frac{\p f}{\p y}, \frac{\p f}{\p z}\>\]

  \item Nested Functions
    \begin{itemize}
      \item If $f$ is a function of $\vect{r}=\<x,y,z\>$, and $x,y,z$ are functions of $t$, then we say $x,y,z$ are itermediate variables and may consider the following composed function of $t$:
        \[f(\vect{r}(t))=f(x(t),y(t),z(t))\]
      \item If $f$ is a function of $\vect{r}=\<x,y,z\>$, and $x,y,z$ are functions of $\vect{s}=\<t,u,v\>$, then we say $x,y,z$ are itermediate variables and may consider the following composed function of $t,u,v$:
        \[f(\vect{r}(\vect{s}))=f(x(t,u,v),y(t,u,v),z(t,u,v))\]
    \end{itemize}
  
  \item Chain Rule
    \begin{itemize}
      \item For functions of the form $f(\vect{r}(t))=f(x(t),y(t),z(t))$:
        \[
          \frac{df}{dt}=\nabla{f}\cdot\frac{d\vect{r}}{dt}=\frac{\p f}{\p x}\frac{dx}{dt}+\frac{\p f}{\p y}\frac{dy}{dt}+\frac{\p f}{\p z}\frac{dz}{dt}
        \]
      \item For functions of the form $f(\vect{r}(\vect{s}))=f(x(t,u,v),y(t,u,v),z(t,u,v))$:
        \[
          \frac{\p f}{\p t}=\nabla{f}\cdot\frac{\p\vect{r}}{\p t}=\frac{\p f}{\p x}\frac{\p x}{\p t}+\frac{\p f}{\p y}\frac{\p y}{\p t}+\frac{\p f}{\p z}\frac{\p z}{\p t}
        \]
    \end{itemize}
    
  \item Differentiation by Substitution
  
    \begin{itemize}
    \item The multi-variable Chain Rule can be avoided by ``plugging in'' functions and using single-variable calculus.
    \end{itemize}

  \newpage

  \item Total Derivative
    \begin{itemize}
    \item If $f$ is a function of $x,y,z$, and $y,z$ are also functions of $x$, then
      \[
        \frac{df}{dx} = \nabla f \cdot \frac{d\vect{r}}{dx} = 
        \frac{\partial f}{\partial x} + \frac{\partial f}{\partial y}\frac{dy}{dx} + \frac{\partial f}{\partial z}\frac{dz}{dx} 
      \]
    \end{itemize}
    
  \item Implicit Differentiation
    \begin{itemize}
    \item If $f(x,y)=c$ defines $y$ as a function of $x$, then
      \[\ds\frac{dy}{dx} = -\frac{\p f/\p x}{\p f/\p y} = -\frac{f_x}{f_y}\]
    \end{itemize}

  \item Tree Diagram for the Chain Rule
    \begin{itemize}
      \item The tree diagram for the chain rule can be used to generate the chain rule.
      \item It also holds for multiple levels of intermediate variables.
    \end{itemize}
    
  % \item \textbf{Suggested Exercises for 12.4:}
  
  %   \begin{itemize}
  %   \item Finding $\frac{dw}{dt}$ for $w=f(x(t),y(t),z(t))$: 1-6
  %   \item Finding partial derivatives for compositions of multi-variable functions: 7-12, 33-38
  %   \item Using partial derivatives for implicit differentiation: 25-28
  %   \end{itemize}

\end{itemize}

\newpage

\centerline{\bf 14.6 Directional Derivatives and the Gradient Vector}

\begin{itemize}

\item Directional Derivative

  \begin{itemize}
    \item The \textbf{directional derivative} of $f$ for the unit vector $\vect{u}$ is 
      \[ 
        D_{\vect{u}}f = \nabla f \cdot \vect{u}
      \]

    \item The maximum value of $D_{\vect{u}}f$ at a fixed point $P_0$ is $|\nabla f(P_0)|$, which occurs when $\vect{u}=\frac{\nabla f(P_0)}{|\nabla f(P_0)|}$.
  \end{itemize}
  
\item Normal Vector to Level Curves and Surfaces
  \begin{itemize}
    \item The gradient vectors $\nabla f$ are normal vectors to the level curves $f(x,y)=k$ for every $(x,y)$ in the domain of $f$.
    \item The gradient vectors $\nabla f$ are normal vectors to the level surfaces $f(x,y,z)=k$ for every $(x,y,z)$ in the domain of $f$.
  \end{itemize}

% \item Gradient Rules

%     \begin{enumerate}
%     \item Constant Multiple Rule
%     \[\nabla(kf)=k\nabla f\]
%     \item Sum Rule
%     \[\nabla(f+g)=\nabla f+\nabla g\]
%     \item Difference Rule
%     \[\nabla(f-g)=\nabla f-\nabla g\]
%     \item Product Rule
%     \[\nabla(fg)=g(\nabla f)+f(\nabla g)\]
%     \item Quotient Rule
%     \[\nabla\left(\frac{f}{g}\right)=\frac{g(\nabla f)-f(\nabla g)}{g^2}\]
%     \end{enumerate}

% \item \textbf{Suggested Exercises for 12.5:}

%   \begin{itemize}
%   \item Finding $\nabla f$ at a point: 1-8
%   \item Finding directional derivatives: 9-16
%   \item Finding the direction of maximal/minimal rate of change: 17-22
%   \item Finding the direction of no instantaneous change: 27-28
%   \end{itemize}

\end{itemize}

\newpage

% \centerline{\bf 12.6 Tangent Planes and Differentials}

% \begin{itemize}

%   \item Normal Vector to a Level Surface
%     \begin{itemize}
%       \item 
%         $\nabla f$ is normal to the level surface $f(x,y,z)=c$ for every point $(x,y,z)$ in the domain of $f$.
%     \end{itemize}
    
%   \item Normal Vector to the Surface $z=f(x,y)$
%     \begin{itemize}
%       \item 
%         If $g(x,y,z)=f(x,y)-z$, then \[\nabla g = \<f_x,f_y,-1\>\] is normal to the surface $z=f(x,y)$ for every point $(x,y)$ in the domain of $f$.
%     \end{itemize}
    
%   \item Tangent Line to Curve of Intersection of Two Surfaces
%     \begin{itemize}
%       \item 
%         If $P_0$ is a point on two surfaces with normal vectors $\vect{n_1},\vect{n_2}$, then the tangent line to the curve of intersection is given by
%         \[
%           \vect{r}(t) = \vect{P_0} + t(\vect{n_1} \times \vect{n_2})
%         \]
%     \end{itemize}
    
%   \item \textbf{Suggested Exercises for 12.6:}
  
%     \begin{itemize}
%     \item Finding tangent planes \& normal lines to surfaces of the form $f(x,y,z)=c$: 1-8
%     \item Finding tangent planes \& normal lines to surfaces of the form $z=f(x,y)$: 9-12
%     \item Finding tangent lines to curves of intersection: 13-18
%     \end{itemize}
  
% \end{itemize}

% \newpage

\centerline{\bf 14.7 Maximum and Minimum Values}

\begin{itemize}

  \item Local Maximum and Minimum Values
  
    \begin{itemize}
    \item Let $f$ be a function of many variables defined near the point $P_0$.
      \begin{itemize}
      \item $f$ has a \textbf{local maximum} $f(P_0)$ at $P_0$ if $f(P_0)$ is the largest value of $f$ near $P_0$
      \item $f$ has a \textbf{local minimum} $f(P_0)$ at $P_0$ if $f(P_0)$ is the smallest value of $f$ near $P_0$
      \end{itemize}
    \end{itemize}

  \item Critical Points
    \begin{itemize}
      \item
        If $P_0$ is a point in the domain of $f$ and
          \[
            \nabla f(P_0) = 0 \text{ or } \nabla f(P_0) \text{ DNE}
          \]
        then $P_0$ is called a \textbf{critical point}.
      \item
        Critical points occur when the tangent plane is horizontal or DNE.
      \item
        The local maximum and minimum values of a function always occur at critical points.
    \end{itemize}
    
  \item Saddle Points
  
    \begin{itemize}
      \item 
        Not every critical point gives a local extreme value.
      \item 
        The \textbf{saddle points} of $f$ are the critical points which don't yield local extreme values.
    \end{itemize}
    
  \item Discriminant Function
    \begin{itemize}
      \item The \textbf{discriminant} of $f$ with variables $x,y$ is the function
        \[
          f_D = \begin{array}{|cc|}f_{xx}&f_{xy}\\f_{yx}&f_{yy}\end{array} = f_{xx}f_{yy} - f_{xy}^2
        \]
    \end{itemize}

  \item Second Derivative Test for Local Extreme Values of $f(x,y)$

    Let $(a,b)$ be a critical point of of $f$ where $\nabla f$ is defined.
    \begin{itemize}
      \item If $f_D(a,b)>0$ and $f_{xx}(a,b)<0$, then $f(a,b)$ is a local maximum.
      \item If $f_D(a,b)>0$ and $f_{xx}(a,b)>0$, then $f(a,b)$ is a local minimum.
      \item If $f_D(a,b)<0$, then $f$ has a saddle point at $(a,b)$.
      \item If $f_D(a,b)=0$, then the test is inconclusive.
    \end{itemize}

  \newpage

  \item Absolute Maximum and Minimum Values
  
    \begin{itemize}
    \item Let $f$ be a function of many variables.
      \begin{itemize}
      \item $f$ has an \textbf{absolute maximum} $f(P_0)$ at $P_0$ if $f(P_0)$ is the largest value in the range of $f$
      \item $f$ has an \textbf{absolute minimum} $f(P_0)$ at $P_0$ if $f(P_0)$ is the smallest value in the range of $f$
      \end{itemize}
    \item Every continuous function of many variables with a closed and bounded domain has an absolute maximum and minimum value.
    \end{itemize}
    
  \item Finding Absolute Max/Min of $f(x,y)$ on a Closed and Bounded Region $D$
    \begin{itemize}
      \item The following points are candidates for giving the absolute extrema:
        \begin{itemize}
          \item Critical points of $f$ within $D$.
          \item Critical points for a function which gives part of the boundary of $D$.
          \item Corners of $D$.
        \end{itemize}
      \item Plug each of these into $f(x,y)$. The largest of these is the absolute maximum, and the smallest of these is the absolute minimum.
    \end{itemize}
    
  % \item \textbf{Suggested Exercises for 12.7:}
  
  %   \begin{itemize}
  %   \item Finding local max/min and saddle points: 1-30
  %   \item Finding absolute max/min: 31-36
  %   \end{itemize}
  
\end{itemize}
  
  \newpage
  
  \centerline{\bf 14.8 Lagrange Multipliers}
  
  \begin{itemize}
  
  \item The Method of Lagrange Multipliers
  
    \begin{itemize}
    \item 
      The \textbf{Method of Lagrange Multipliers} says that if $f$ is a function of many variables which has an absolute max/min value on the restriction $g(P)=k$ where $\nabla g \not= 0$, then the absolute max/min occurs at a point $P$ where 
      \[
        \nabla f(P)=\lambda \nabla g(P) \text{ and } g(P)=k
      \] 
      for some real number $\lambda$.

    \item 
      If two constraints $g(P)=k$ and $h(P)=l$ are given, then the absolute max/min occurs where
      \[
        \nabla f(P)=\lambda \nabla g(P) + \mu \nabla h(P) \text{ and } g(P)=k \text{ and } h(P)=l
      \]
      for some real numbers $\lambda,\mu$.

    \end{itemize}
    
  % \item \textbf{Suggested Exercises for 12.8:}
  
  %   \begin{itemize}
  %   \item Finding absolute extrema using the Method of Lagrange Multipliers: 1-30
  %   \end{itemize}
    
  \end{itemize}
  
  \newpage
  
\centerline{\bf 15.1 Double Integrals over Rectangles}
  
  \begin{itemize}

  \item Double Integral
    \begin{itemize}
    \item We define the \textbf{double integral} of a function $f(x,y)$ over a region $R$ to be 
      \[
        \iint_R f(x,y)\d{A} = \lim_{n\to\infty}\sum_{i=1}^n f(x_{n,i},y_{n,i})\Delta A_{n,i}
      \]
    where for each positive integer $n$ we've defined a way to partition $R$ into $n$ pieces 
      \[
        \Delta R_{n,1},\Delta R_{n,2},\dots,\Delta R_{n,n}
      \]
    where $\Delta R_{n,i}$ has area $\Delta A_{n,i}$, contains the point $(x_{n,i},y_{n,i})$, and \[\lim_{n\to\infty} \max(\Delta A_{n,i}) = 0\]
    \item Since for $f(x,y)\geq 0$,
      \[
        \sum_{i=1}^n f(x_{n,i},y_{n,i})\Delta A_{n,i}
      \]
      is an approximation of the volume under $z=f(x,y)$ and over $R$, the double integral is used to define the precise volume.
    \item If $f$ is not always positive, then the double integral represents \textbf{net volume}: volume above the $xy$-plane minus volume below the $xy$-plane.
    \end{itemize}

  \item Midpoint Rule for Approximating Rectangular Double Integrals
    \begin{itemize}
      \item For the rectangle
      \[
        R: a\leq x\leq b, c\leq y\leq d
      \]
      we may approximate the double integral by partitioning the rectangle into a grid of $m\times n$ rectangular pieces all with area $\Delta A$ and evaluating:
      \[
        \iint_R f(x,y)\d{A} \approx \sum_{i=1}^m\sum_{j=1}^n f(\overline{x_i},\overline{y_j}) \Delta A
      \]
      where $(\overline{x_i},\overline{y_j})$ is the midpoint of the $i\times j$ rectangle.
    \end{itemize}

  \end{itemize}

\newpage

\centerline{\bf 15.2 Iterated Integrals}

  \begin{itemize}

  \item Volume as Integral of Area

    \begin{itemize}
    \item If $A(x)$ is the area of a solid's cross-section, then the solid's volume is
      \[
        V  = \int_a^b A(x)\d{x}
      \]
    \end{itemize}
  
  \item Iterated Integrals over Rectangles
  
    \begin{itemize}
    \item A double integral over a rectangle
      \[
        R: a\leq x\leq b, c\leq y\leq d
      \]
        can be expressed as the \textbf{iterated integrals}:
      \[
        \iint_R f(x,y)\d{A} = \int_a^b\int_c^d f(x,y)\d{y}\d{x} = \int_c^d\int_a^b f(x,y)\d{x}\d{y}
      \]
    \end{itemize}
    
  % \item \textbf{Suggested Exercises for 13.1:}
  
  %   \begin{itemize}
  %   \item Evaluating iterated integrals with constant bounds: 1-12
  %   \item Evaluating double integrals over rectangles: 13-28
  %   \end{itemize}
    
  \end{itemize}
  
  \newpage
  
  \centerline{\bf 15.3 Double Integrals over General Regions}
  
  \begin{itemize}
  
  \item Double Integrals over Nonrectangular Regions
  
    \begin{itemize}
    \item For \textbf{Type I} regions which may be expressed as \[R: a\leq x\leq b, g_1(x)\leq y\leq g_2(x)\] a double integral over $R$ may be expressed as the iterated integral:
      \[
        \iint_R f(x,y)\d{A} = \int_a^b\int_{g_1(x)}^{g_2(x)} f(x,y)\d{y}\d{x}
      \]
    \item For \textbf{Type II} regions which may be expressed as \[R: h_1(y)\leq x\leq h_2(y), a\leq y\leq b\] a double integral over $R$ may be expressed as the iterated integral:
      \[
        \iint_R f(x,y)\d{A} = \int_a^b\int_{h_1(y)}^{h_2(y)} f(x,y)\d{x}\d{y}
      \]
    \end{itemize}

  \item Finding Limits of Integration
      \begin{enumerate}
      \item Sketch the region and label bounding curves
      \item Determine if the region is Type I or Type II by identifying (I) bottom/top curves $y=g_1(x),y=g_2(x)$ or (II) left/right curves $x=h_1(y),x=h_2(y)$.
      \end{enumerate}
      For Type I:
      \begin{enumerate}
      \setcounter{enumi}{2}
      \item Use the leftmost and rightmost $x$-values in the region $a,b$ to complete the iterated integral:
        \[
          \iint_R f(x,y)\d{A} = \int_a^b\int_{g_1(x)}^{g_2(x)}f(x,y)\d{y}\d{x}
        \]
      \end{enumerate}
      For Type II:
      \begin{enumerate}
      \setcounter{enumi}{2}
      \item Use the bottommost and topmost $y$-values in the region $c,d$ to complete the iterated integral:
        \[
          \iint_R f(x,y)\d{A} = \int_c^d\int_{h_1(y)}^{h_2(y)}f(x,y)\d{x}\d{y}
        \]
      \end{enumerate}
  
  \item Swapping Variables of Integration
  
    \begin{itemize}
    \item You can only swap the order of integration of an iterated integral by drawing the region and reinterpreting it as a region of the opposite Type.
    \end{itemize}

  % \item Properties of Double Integrals
  %     \begin{enumerate}
  %     \item Zero Integral
  %       \[\iint\limits_R 0\,dA = 0\]
  %     \item Constant Multiple
  %       \[\iint\limits_R cf(x,y)\,dA = c\iint\limits_R f(x,y)\,dA\]
  %     \item Sum/Difference
  %       \[\iint\limits_R f(x,y)\pm g(x,y)\,dA=\iint\limits_R f(x,y)\,dA\pm\iint\limits_R g(x,y)\,dA\]
  %     \item Domination
      
  %     If $f(x,y)\leq g(x,y)$ for all $(x,y)\in R$, then
        % \[\iint\limits_R f(x,y)\,dA \leq \iint\limits_R g(x,y)\,dA\]
    \item Additivity
    
    If $R$ can be split into two regions $R_1,R_2$, then
      \[\iint\limits_R f(x,y)\,dA = \iint\limits_{R_1} f(x,y)\,dA + \iint\limits_{R_2} f(x,y)\,dA\]
      % \end{enumerate}

  \item Average Value of Two-Variable Functions
    \begin{itemize}
      \item The average value of a two-variable function $f$ over a region $R$ is defined to be
        \[
          \frac{1}{\text{Area of }R}\iint_R f(x,y)\d{A}
        \]
    \end{itemize}

  \item Area as a Double Integral
    
    \begin{itemize}
    \item The area of a region $R$ in the plane is \[A = \iint\limits_R\,dA=\iint\limits_R 1\,dA\]
    \end{itemize}
    
  % \item \textbf{Suggested Exercises for 13.2:}
  
  %   \begin{itemize}
  %   \item Evaluating nonrectangular double integrals: 1-6, 11-14
  %   \item Finding limits of integration: 7-10, 33-44
  %   \item Swapping order of integration: 25-32
  %   \end{itemize}
    
  \end{itemize}
  
  \newpage
  
  % \centerline{\bf 13.3 Area by Double Integration}
  
  % \begin{itemize}
  
  % \item Areas of Regions in the Plane
    
  %   \begin{itemize}
  %   \item The area of a region $R$ in the plane is \[A = \iint\limits_R\,dA=\iint\limits_R 1\,dA\]
  %   \end{itemize}
    
  % \item Average Value of a Function of Two Variables
  
  %   \begin{itemize}
  %   \item The average value of $f(x,y)$ over the region $R$ is defined to be \[\textrm{Avg Val} = \frac{1}{\textrm{area of }R}\iint\limits_R f(x,y)\,dA\]
  %   \end{itemize}
    
  % \item \textbf{Suggested Exercises for 13.3:}
  
  %   \begin{itemize}
  %   \item Finding areas of regions: 1-8
  %   \item Finding average values of functions: 15-18
  %   \end{itemize}
    
  % \end{itemize}
  
  % \newpage
  
  \centerline{\bf 15.7 Triple Integrals}
  
  \begin{itemize}

  \item Triple Integral
    \begin{itemize}
    \item We define the \textbf{triple integral} of a function $f(x,y,z)$ over a solid $D$ to be 
        \[
          \iiint_D f(x,y,z)\d{V} = \lim_{n\to\infty}\sum_{i=1}^n f(x_{n,i},y_{n,i},z_{n,i})\Delta V_{n,i}
        \]
      where for each positive integer $n$ we've defined a way to partition $D$ into $n$ pieces 
        \[
          \Delta D_{n,1},\Delta D_{n,2},\dots,\Delta D_{n,n}
        \]
      where $\Delta D_{n,i}$ has volume $\Delta V_{n,i}$, contains the point $(x_{n,i},y_{n,i},z_{n,i})$, and \[\lim_{n\to\infty} \max(\Delta V_{n,i}) = 0\]
    \end{itemize}

  \item Iterated Integral for Rectangular Boxes
    \begin{itemize}
      \item The triple integral over the rectangular box
        \[
          D: a_1\leq x\leq a_2, b_1\leq y\leq b_2, c_1\leq z\leq c_2
        \]
        can be expressed as the \textbf{iterated integrals}:
      \[
        \iiint_D f(x,y,z)\d{V} = \int_{a_1}^{a_2}\int_{b_1}^{b_2}\int_{c_1}^{c_2} f(x,y,z)\d{z}\d{y}\d{x}
      \]
      \[
        = \int_{b_1}^{b_2}\int_{c_1}^{c_2}\int_{a_1}^{a_2} f(x,y,z)\d{x}\d{z}\d{y} 
        = \int_{a_1}^{a_2}\int_{c_1}^{c_2}\int_{b_1}^{b_2} f(x,y,z)\d{y}\d{z}\d{x} 
        = \dots
      \]
    \end{itemize}

  \item Iterated Integral for Generated Solids
    \begin{itemize}
      \item If the solid $D$ is determined by the bottom/top surfaces
        \[
          h_1(x,y)\leq z\leq h_2(x,y)
        \]
      and has shadow $R$ in the $xy$-plane, then a triple integral over $D$ can be expressed as:
        \[
          \iiint_D f(x,y,z)\d{V} = \iint_R\left[ \int_{h_1(x,y)}^{h_2(x,y)} f(x,y,z) \d{z}  \right]\d{A}
        \]
      \item In general:
        \[
          \iiint_D f(x,y,z)\d{V} = \iint_R\left[ \int_{\text{bottom surface}}^{\text{top surface}} f(x,y,z) \d{\square}  \right]\d{A}
        \]
      where $\square$ is chosen from $x,y,z$ to be the ``up'' orientation.
    \end{itemize}

  \item Additivity
  
  If $D$ can be split into two regions $D_1,D_2$, then
    \[\iint\limits_D f(x,y,z)\d{V} = \iint\limits_{D_1} f(x,y,z)\d{V} + \iint\limits_{D_2} f(x,y,z)\d{V}\]

  \item Average Value of Three-Variable Functions
    \begin{itemize}
      \item The average value of a three-variable function $f$ over a solid $D$ is defined to be
        \[
          \frac{1}{\text{Volume of }D}\iiint_R f(x,y,z)\d{V}
        \]
    \end{itemize}
  
  \item Volume as a Triple Integral
    
    \begin{itemize}
    \item The volume of a solid $D$ in space is \[V = \iiint\limits_D\,dV=\iiint\limits_D 1\,dV\]
    \end{itemize}
    
  \end{itemize}
  
  \newpage
  
  \centerline{\bf 15.10 Change of Variables in Multiple Integrals}
  
  \begin{itemize}
  
  \item Transformations
  
    \begin{itemize}
    \item Two similar regions in 2D space can be transformed by a ``nice'' pair of functions \[\vect{r}(u,v) = \vect{r}(\vect{s}) = \<x(\vect{s}),y(\vect{s})\> = \<x(u,v),y(u,v)\>\] that map points in a $uv$ plane to the $xy$ plane.
    \item Two similar solids in 3D space can be transformed by a ``nice'' triple of functions \[\vect{r}(u,v,w) = \vect{r}(\vect{s}) = \<x(\vect{s}),y(\vect{s}),z(\vect{s})\> = \<x(u,v,w),y(u,v,w),z(u,v,w)\>\] that map points in a $uvw$ space to the $xyz$ space.
    \end{itemize}
  
  \item The Jacobian
  
    \begin{itemize}
    \item The Jacobian of a 2D transformation given by $\vect{r}(u,v)$ is the determinant
\[
\vect{r}_J(u,v) = \frac{\p (x,y)}{\p (u,v)} = \frac{\p\vect{r}}{\p\vect{s}} =
\begin{array}{|c c|}
\frac{\p x}{\p u} & \frac{\p x}{\p v} \\
\frac{\p y}{\p u} & \frac{\p y}{\p v} \\
\end{array}
\]
    \item The Jacobian of a 3D transformation given by $\vect{r}(u,v,w)$ is the determinant
\[
\vect{r}_J(u,v,w) = \frac{\partial (x,y,z)}{\partial (u,v,w)} = \frac{\p\vect{r}}{\p\vect{s}} =
\begin{array}{|c c c|}
\frac{\partial x}{\partial u} & \frac{\partial x}{\partial v} & \frac{\partial x}{\partial w} \\
\frac{\partial y}{\partial u} & \frac{\partial y}{\partial v} & \frac{\partial y}{\partial w} \\
\frac{\partial z}{\partial u} & \frac{\partial z}{\partial v} & \frac{\partial z}{\partial w} \\
\end{array}
\]
    \end{itemize}
    
  \item 2D Substitution
  
    \begin{itemize}
    \item Suppose that the region $R$ in the $xy$-plane is the result of applying the transformation $\vect{r}(u,v)$ to the region $G$ in the $uv$-plane.
    \item Then it follows that \[\iint\limits_R f(x,y)\,dx\,dy = \iint\limits_G f(x(u,v),y(u,v))|\vect{r}_J(u,v)|\,du\,dv\]
    \end{itemize}

  \newpage

  \item Unit Square and Triangle

    \begin{itemize}
      \item The unit square in the $uv$ plane with vertices $(0,0)$, $(1,0)$, $(1,1)$, and $(0,1)$ is useful for substitution problems involving parallelograms.
      \item The unit triangle in the $uv$ plane with vertices $(0,0)$, $(1,0)$, and $(1,1)$ is useful for substitution problems involving triangles.
    \end{itemize}
    
  \item 3D Substitution
  
    \begin{itemize}
    \item Suppose that the solid $D$ in $xyz$ space is the result of applying the transformation $\vect{r}(u,v,w)$ to the region $H$ in $uvw$ space.
    \item Then it follows that \[\iiint\limits_D f(x,y,z)\,dx\,dy\,dz \]\[= \iiint\limits_H f(x(u,v,w),y(u,v,w),z(u,v,w))|\vect{r}_J(u,v,w)|\,du\,dv\,dw\]
    \end{itemize}
        
  % \item \textbf{Suggested Exercises for 13.8:}
  
  %   \begin{itemize}
  %   \item 2D Jacobians, Transformations, and substitutions: 1-10
  %   \end{itemize}
    
  \end{itemize}
  
\newpage

\centerline{\bf 15.4 Double Integrals in Polar Coordinates}
  
  \begin{itemize}
  
  \item Integrating over Regions expressed using Polar Coordinates
    
    \begin{itemize}
    \item The polar coordinate transformation \[\vect{r}(r,\theta) = \<r\cos\theta, r\sin\theta\>\] from polar $G$ into Cartesian $R$ yields \[\iint\limits_R f(x,y)\, dA = \iint\limits_G f(r\cos\theta,r\sin\theta)\,r\dvar{r}\dvar{\theta}\]
    \end{itemize}
      
  % \item \textbf{Suggested Exercises for 13.4:}
  
  %   \begin{itemize}
  %   \item Changing Cartesian integrals to polar integrals: 1-16
  %   \item Finding integrals over polar regions: 17-22
  %   \end{itemize}
    
  \end{itemize}
  
% \newpage

\hr

\centerline{\bf 15.8 Triple Integrals in Cylindrical Coordinates}
  
  \begin{itemize}
  
  \item Cylindrical Coordinates
    \begin{itemize}
    \item The cylindrical coordinate transformation \[\vect{r}(r,\theta,z) = \<r\cos\theta, r\sin\theta, z\>\] from cylindrical $H$ into Cartesian $D$ yields \[\iiint\limits_D f(x,y,z)\d{V} = \iiint\limits_H f(r\cos\theta,r\sin\theta,z)\,r\d{r}\d{\theta}\d{z}\]
    \end{itemize}
      
  % \item \textbf{Suggested Exercises for 13.7:}
  
  %   \begin{itemize}
  %   \item Cylindrical coordinate integrals: 1-20
  %   \item Finding integrals over polar regions: 21-38
  %   \end{itemize}
  
  \end{itemize}

% \newpage

\hr

\centerline{\bf 15.9 Triple Integrals in Spherical Coordinates}

  \begin{itemize}
  
  \item Spherical Coordinates
    \begin{itemize}
    \item The spherical coordinate transformation \[\vect{r}(\rho,\phi,\theta)=\<\rho\sin\phi\cos\theta, \rho\sin\phi\sin\theta, \rho\cos\phi\>\] from spherical $H$ into Cartesian $D$ yields \[\iiint\limits_D f(x,y,z)\, dV = \iiint\limits_H f(\rho\sin\phi\cos\theta,\rho\sin\phi\sin\theta,\rho\cos\phi) \,\rho^2\sin\phi\dvar{\rho}\dvar{\phi}\dvar{\theta} \]
    \end{itemize}
  \end{itemize}
  
\newpage

\centerline{\bf 16.1 Vector Fields}

  \begin{itemize}

  \item Vector Fields
    \begin{itemize}
    \item A \textbf{vector field} assigns a vector to each point in 2D or 3D space.
    \end{itemize}
    \[
      \vect{F}=\vect{F}(\vect{r})=\vect{F}(x,y)=\<P(x,y),Q(x,y)\>=\<P(\vect{r}),Q(\vect{r})\>=\<P,Q\>
    \]
    \[
      \vect{F}=\vect{F}(\vect{r})=\vect{F}(x,y,z)=\<P(x,y,z),Q(x,y,z),R(x,y,z)\>=\<P(\vect{r}),Q(\vect{r}),R(\vect{r})\>=\<P,Q,R\>
    \]

  \item Gradient Vector Field
    \begin{itemize}
    \item The gradient vector field $\nabla f(x,y)=\<f_x(x,y),f_y(x,y)\>$ assigns vectors whose directions are normal to level curves and whose magnitudes are equal to the maximal directional derivative at the point.
    \item The gradient vector field $\nabla f(x,y,z)=\<f_x(x,y,z),f_y(x,y,z),f_z(x,y,z)\>$ assigns vectors whose directions are normal to level surfaces and whose magnitudes are equal to the maximal directional derivative at the point. 
    \end{itemize}

  \end{itemize}

\newpage

\centerline{\bf 16.2 Line Integrals}

  \begin{itemize}

  \item Common Curve Parametrizations

    \begin{itemize}
      \item A line segment beginning at $P_0$ and ending at $P_1$
        \[
          \vect{r}(t) = \vect{P_0} + t(\vect{P_1}-\vect{P_0}), 0\leq t\leq 1
        \]
      \item A circle centered at the origin with radius $a$
        \[
          \vect{r}(t) = \<a\cos t,a\sin t\>, 0\leq t\leq 2\pi \text{ (counter-clockwise)}
        \]
        \[
          \vect{r}(t) = \<a\sin t,a\cos t\>, 0\leq t\leq 2\pi \text{ (clockwise)}
        \]
      \item A planar curve given by $y=f(x)$ from $(x_0,y_0)$ to $(x_1,y_1)$
        \[
          \vect{r}(t) = \<t,f(t)\>, x_0\leq t\leq x_1 \text{ (for } x_0\leq x_1 \text{)}
        \]
        \[
          \vect{r}(t) = \<-t,f(-t)\>, -x_0\leq t\leq -x_1 \text{ (for } x_0\leq x_1 \text{)}
        \]
    \end{itemize}
  
  \item Line Integrals with Respect to Arclength
  
    \begin{itemize}
    \item We define the \textbf{line integral with respect to arclength} of a function of many variables $f(\vect{r})$ along a curve $C$ to be 
      \[
        \int_C f(\vect{r})\d{s} = \lim_{n\to\infty}\sum_{i=1}^n f(\vect{r}_{n,i})\Delta s_{n,i}
      \]
    where for each positive integer $n$ we've defined a way to partition $C$ into $n$ pieces 
      \[
        \Delta C_{n,1},\Delta C_{n,2},\dots,\Delta C_{n,n}
      \]
    where $\Delta C_{n,i}$ has length $\Delta s_{n,i}$, contains the position vector $\vect{r}_{n,i}$, and \[\lim_{n\to\infty} \max(\Delta s_{n,i}) = 0\]
    \item If $\vect{r}(t)$ is a parametrization of $C$ for $a \leq t \leq b$, then 
      \[
        \int\limits_C f(\vect{r})\dvar{s}=\int_{t=a}^{t=b} f(\vect{r}(t))\frac{ds}{dt}\d{t}
      \]
    \end{itemize}

  \newpage

  \item Line Integrals with Respect to Variables

    \begin{itemize}
    \item Similarly, we can find the \textbf{line integral with respect to a variable} for a function of many variables $f(\vect{r})$ along a curve $C$:
      \[
        \int\limits_C f(\vect{r})\dvar{x}=\int_{t=a}^{t=b} f(\vect{r}(t))\frac{dx}{dt}\d{t}
      \]
    \item Similar defintions hold for $y,z$.
    \end{itemize}

  \item Line Integrals of Vector Fields

    \begin{itemize}
    \item The \textbf{line integral of a vector field} is defined to be the line integral with respect to arclength of the dot product of the vector field $\vect{F}(\vect{r})=\<P(\vect{r}),Q(\vect{r}),R(\vect{r})\>$ with the unit tangent vector $\vect{T}(\vect{r})$ to the curve.
      \[
        \int_C \vect{F}(x,y,z)\cdot\vect{T}(x,y,z)\d{s}
      \]
    \item There are several ways to write and evaluate line integrals of vector fields:
      \[
        \int_C \vect{F}\cdot\vect{T}\d{s} = \int_C \vect{F}\cdot\d{\vect{r}} = \int_C \<P,Q,R\>\cdot\<\d x,\d y,\d z\>
      \]
      \[
        = \int_C P\d{x}+Q\d{y}+R\d{z} = \int_a^b \left(P(\vect{r}(t))\frac{dx}{dt}+Q(\vect{r}(t))\frac{dy}{dt}+R(\vect{r}(t))\frac{dz}{dt}\right)\d{t}
      \]
      \[
        = \int_a^b \vect{F}(\vect{r}(t))\cdot\frac{\d{\vect{r}}}{dt}\d{t}
      \]
    \end{itemize}
    
  \item Additivity

  Let $C_1+C_2$ represent the curve taken by moving along $C_1$ followed by moving along $C_2$.

   \[\int\limits_{C_1+C_2} f(\vect{r})\d{s}=\int\limits_{C_1}f(\vect{r})\d{s}+\int\limits_{C_2}f(\vect{r})\d{s}\]

   \[\int\limits_{C_1+C_2} f(\vect{r})\d{x}=\int\limits_{C_1}f(\vect{r})\d{x}+\int\limits_{C_2}f(\vect{r})\d{x}\]

   \[\int\limits_{C_1+C_2} \vect{F}\cdot\d{\vect{r}}=\int\limits_{C_1}\vect{F}\cdot\d{\vect{r}}+\int\limits_{C_2}\vect{F}\cdot\d{\vect{r}}\]

  \item Effects of Curve Orientation

  Let $-C$ represent the curve taken by moving along $C$ in the opposite direction.

  \[\int\limits_{C} f(\vect{r})\d{s} = +\int\limits_{-C} f(\vect{r})\d{s}\]

  \[\int\limits_{C} f(\vect{r})\d{x} = -\int\limits_{-C} f(\vect{r})\d{x}\]

  \[\int\limits_{C} \vect{F}\cdot\d{\vect{r}} = -\int\limits_{-C} \vect{F}\cdot\d{\vect{r}}\]

  \item Work

    \begin{itemize}
      \item If $\vect{F}$ is a vector field representing the force applied to an object as it is moved over a smooth curve $C$, then the \textbf{work} done by the force over that curve is given by
        \[
          \int_C \vect{F}\cdot\d{\vect{r}} = \int_a^b \vect{F}(\vect{r}(t))\cdot\frac{\d{\vect{r}}}{dt}\d{t}
        \]
    \end{itemize}
        
  % \item \textbf{Suggested Exercises for 14.1:}
  
  %   \begin{itemize}
  %   \item Identifying vector equations for graphs: 1-8
  %   \item Evaluating line integrals: 9-22
  %   \end{itemize}
    
  \end{itemize}

\newpage

\centerline{\bf 16.3 The Fundamental Theorem for Line Integrals}

  \begin{itemize}

  \item The Fundamental Theorem
    \begin{itemize}
      \item If $C$ is any smooth curve beginning at the point $A$ and ending at the point $B$, then
        \[
          \int_C \nabla f\cdot \d{\vect{r}} = \left[f\right]_A^B = f(B)-f(A)
        \]
      \item If $C$ is any smooth curve which is \textbf{closed} (begins and ends at the same point), then
        \[
          \int_C \nabla f\cdot \d{\vect{r}} = 0
        \]
    \end{itemize}

  \item Conservative Fields
    \begin{itemize}
      \item We say $\vect{F}=\<M,N,P\>$ is a conservative field if there is a \textbf{potential function} $f$ such that $\nabla f = \vect{F}$.
      \item Line integrals of conservative fields are said to be path independent since for any curve $C$ beginning at $A$ and ending at $B$:
        \[
          \int_C \vect{F}\cdot \d{\vect{r}} = \int_C \nabla f\cdot \d{\vect{r}} = \left[f\right]_A^B = f(B)-f(A)
        \]
      \item We can prove a field is conservative by finding its potential function or showing it satisfies the \textbf{Component Test}:
        \[
          \frac{\p P}{\p y}=\frac{\p Q}{\p x}, \frac{\p Q}{\p z}=\frac{\p R}{\p y}, \frac{\p R}{\p x}=\frac{\p P}{\p z}
        \]
    \end{itemize}

  \end{itemize}
  
\newpage

\centerline{\bf 16.4 Green's Theorem}

  \begin{itemize}
    \item Green's Theorem
      \begin{itemize}
        \item Let $C$ be the boundary of the region $R$ oriented counter-clockwise, and $\vect{F}(x,y)$ be a two-dimensional vector field.
          \[
            \int_C \vect{F}\cdot\d{\vect{r}} = \iint_R \left(\frac{\p Q}{\p x}-\frac{\p P}{\p y}\right)\d{A}
          \]
        \item Due to Green's Theorem, we can find the area of $R$ using a line integral:
          \[ 
            A = \iint_R 1 - 0 \d{A} = \int_C x\d{y} 
          \]
          \[
            A = \iint_R 0 - (-1)\d{A} = \int_C -y\d{x} 
          \]
          \[
            A = \iint_R \frac{1}{2} - \left(-\frac{1}{2}\right) \d{A} = \int_C \frac{1}{2}x\d{y}-\frac{1}{2}y\d{x}
          \]
      \end{itemize}
  \end{itemize}

\newpage

\centerline{\bf 16.5 Curl and Divergence}

  \begin{itemize}
    \item Gradient Operator
      \[
        \nabla = \<\frac{\p}{\p x},\frac{\p}{\p y},\frac{\p}{\p z}\>
      \]

    \item Curl
      \begin{itemize}
        \item The \textbf{curl} of a vector field is another vector field:
          \[
            \curl\vect{F} = \nabla \times \vect{F} = \<\pd{R}{y}-\pd{Q}{z},\pd{P}{z}-\pd{R}{x},\pd{Q}{x}-\pd{P}{y}\>
          \]
        \item By the Component Test, if $\vect{F}$ is conservative, then $\curl\vect{F}=\vect{0}$.
      \end{itemize} 

    \item Divergence
      \begin{itemize}
        \item The \textbf{divergence} of a vector field is the scalar function:
          \[
            \div\vect{F} = \nabla\cdot\vect{F} = \pd{P}{x}+\pd{Q}{y}+\pd{R}{z}
          \]
        \item For any vector field, the divergence of curl is always zero.
          \[
            \div\curl\vect{F} = 0
          \]
      \end{itemize}

      \item Green's Theorem Alternate Forms
        \begin{itemize}
          \item If $\vect{F}$ is a two-dimensional vector field, and $\vect{n}$ is the outward unit normal vector field for a counter-clockwise closed curve $C$:
            \[
              \int_C \vect{F}\cdot\vect{T}\d{s} = \iint_D (\curl\vect{F}) \cdot \veck \d{A}
            \]
            \[
              \int_C \vect{F}\cdot\vect{n}\d{s} = \iint_D \div\vect{F} \d{A}
            \]
        \end{itemize}

  \end{itemize}

\newpage

\centerline{\bf 16.6 Parametric Surfaces and Their Areas}

  \begin{itemize}

    \item Parametric Surface Equations
      \[
        \vect{r}(u,v) = \<x(u,v),y(u,v),z(u,v)\>
      \]

    \item Common Parametric Surfaces
      \begin{itemize}
        \item The plane determined by the point $P_0$ and vectors $\vect{v_1}$ and $\vect{v_2}$ can be parametrized by
          \[
            \vect{r} = \vect{P_0}+u\vect{v_1}+v\vect{v_2}
          \]
        \item The surface $z=f(x,y)$ can be parametrized by
          \[
            \vect{r} = \<x,y,f(x,y)\>
          \]
        \item A surface determined by a cylindrical coordinate equation can be parametrized by substituting into
          \[
            \vect{r} = \<r\cos\theta, r\sin\theta, z\>
          \]
        \item A surface determined by a spherical coordinate equation can be parametrized by substituting into
          \[
            \vect{r} = \<\rho\sin\phi\cos\theta, \rho\sin\phi\sin\theta, \rho\cos\phi \>
          \]
      \end{itemize}

    \item Surface Area
      \begin{itemize}
        \item If $G$ is the region in the $uv$ plane which maps onto the surface $S$ by the parametric equations $\vect{r}(u,v)$, then the surface area of $S$ is:
          \[
            \iint_G |\vect{r}_u\times\vect{r}_v|\d{A}
          \]
        where $\vect{r}_u=\<x_u,y_u,z_u\>$ and $\vect{r}_v=\<x_v,y_v,z_v\>$.
      \end{itemize}

  \end{itemize}

\newpage

\centerline{\bf 16.7 Surface Integrals}

  \begin{itemize}

    \item Surface Integral
      \begin{itemize}
        \item If $G$ is the region in the $uv$ plane which maps onto the surface $S$ by the parametric equations $\vect{r}(u,v)$, then the surface integral of $f(\vect{r})$ along $S$ is:
          \[
            \iint_S f(\vect{r})\d\sigma = \iint_G f(\vect{r}(u,v))|\vect{r}_u\times\vect{r}_v|\d{A}
          \]
      \end{itemize}

    \item Surface Orientation
      \begin{itemize}
        \item The orientation of a surface is determined by a continuous unit normal vector field $\vect{n}$ on the surface.
        \item The M\"obius strip is an example of a non-orientable surface.
      \end{itemize}

    \item Surface Integral of Vector Field
      \begin{itemize}
        \item If $G$ is the region in the $uv$ plane which maps onto the surface $S$ by the parametric equations $\vect{r}(u,v)$, and $\vect{n}$ is the unit normal vector field giving the orientation of $S$, then the surface integral of the vector field $\vect{F}$ along $S$ is:
          \[
            \iint_S \vect{F}\cdot\d{\vec\sigma} = \iint_S \vect{F}\cdot\vect{n}\d\sigma = \iint_G \vect{F}\cdot(\vect{r}_u\times\vect{r}_v)\d{A}
          \]
      \end{itemize}
  \end{itemize}

\newpage

\centerline{\bf 16.8 Stokes' Theorem}
  \begin{itemize}
  \item Stokes' Theorem
    \begin{itemize}
    \item Let $C$ give the counter-clockwise oriented boundary of a surface $S$.
    \[
      \iint_S \curl\vect{F}\cdot\d{\vec\sigma} = \int_C \vect{F}\cdot\d{\vect{r}}
    \]
    \end{itemize}
  \end{itemize}

\hr

\centerline{\bf 16.9 Divergence Theorem}
  \begin{itemize}
  \item Divergence Theorem
    \begin{itemize}
    \item Let $S$ give the outward-oriented boundary surface of the solid $D$.
    \[
      \iint_S \vect{F}\cdot\d{\vec\sigma} = \iiint_D \div{\vect{F}}\d{V}
    \]
    \end{itemize}
  \end{itemize}
  
  % \centerline{\bf 14.2 Vector Fields, Work, Circulation, and Flux}
  
  % \begin{itemize}
  
  % \item Line Integrals with Respect to Variables
  
  %   \begin{itemize}
  %     \item The net projected area of the ribbon with base curve $C$ and height $f(x,y,z)$ with respect to the $x$-axis is given by the \textbf{line integral of $f(x,y,z)$ over $C$ with respect to $x$}: \[\int\limits_C f(x,y,z)\,dx\] (similar for $y$, $z$)
  %     \item Line integrals with respect to variables can be evaluated by finding a parametrization $\vect{r}(t)$ for the curve $C$:
  %       \[
  %         \int\limits_C f(x,y,z)\,dx = \int_a^b f(x(t),y(t),z(t))\frac{dx}{dt}\,dt
  %       \]
  %     \item Such integrals have the property \[\int\limits_{-C} f\dvar{x} = -\int\limits_{C} f\dvar{x}\]
  %   \end{itemize}
  
  % \item Vector Fields
  
  %   \begin{itemize}
  %   \item A \textbf{vector field} is a function \[\vect{F}(x,y,z)=\<M(x,y,z),N(x,y,z),P(x,y,z)\>\] ($\vect{F}=\<M,N,P\>$ for short) which assigns a vector to each point in its domain.
  %   \item Gradient functions $\nabla f=\<f_x(x,y,z),f_y(x,y,z),f_z(x,y,z)\>$ and transformations $\<x(u,v,w),y(u,v,w),z(u,v,w)\>$ are examples of vector fields.
  %   \end{itemize}
  
  % \item Line Integrals of Vector Fields
  
  %   \begin{itemize}
  %   \item The \textbf{line integral of $\vect{F}=\<M,N,P\>$ over $C$} is given by 
  %     \[\int\limits_C \vect{F}\cdot d\vect{r} = \int\limits_C M\,dx + N\,dx + P\,dz\]
  %   gives the sum of the line integrals of each component of $\vect{F}$ with respect to each variable $x,y,z$.
  %   \item These line integrals can be calculated by using parametrizations of $C$: 
  %     \[
  %       \int\limits_C \vect{F}\cdot d\vect{r} = 
  %       \int\limits_C M\,dx + N\,dx + P\,dz = 
  %       \int_a^b \left(M\frac{dx}{dt} + N\frac{dx}{dt} + P\frac{dz}{dt}\right)\,dt 
  %     \]
  %     \[
  %       =
  %       \int_a^b \vect{F}\cdot\vect{v}\,dt =
  %       \int_a^b \vect{F}\cdot\vect{T}\,ds 
  %     \]
  %   \item It follows that \[\int\limits_{-C} \vect{F}\cdot d\vect{r} = - \int\limits_C \vect{F}\cdot d\vect{r}\]
  %   \end{itemize}
    
  % \item Work over a Smooth Curve
  
  %   \begin{itemize}
  %   \item Work is given by the product of force and displacement: \[W = \vect{F} \cdot \vect{D}\]
  %   \item So work over a smooth curve can be approximated by the Riemann sum: \[W \approx \sum_{i=1}^n \vect{F}(x_i,y_i,z_i)\cdot\Delta\vect{r_i}\]
  %   \item We limit this sum to infinity to define work over a smooth curve: \[W = \int\limits_C \vect{F}\cdot d\vect{r}\]
  %   \end{itemize}
    
  % \item Flow
  
  %   \begin{itemize}
  %   \item The \textbf{flow} of a fluid along a curve $C$ is defined to be the line integral \[\textrm{Flow} = \int\limits_C \vect{F}\cdot d\vect{r} \]
  %   \item If $C$ is closed (its starting point and ending point are the same), then the flow is also known as the \textbf{circulation}.
  %   \end{itemize}
    
  % \newpage
    
  % \item Flux (2D)
  
  %   \begin{itemize}
  %   \item The two-dimenstional \textbf{flux} of $\vect{F}$ across $C$ is \[\int\limits_C \vect{F}\cdot\vect{n}\,ds\] where $\vect{n}$ is the outward unit normal vector to $C$.
  %   \item If $C$ is oriented counter-clockwise, then 
  %     \[
  %       \int\limits_C \vect{F}\cdot\vect{n}\dvar{s} = \int\limits_C M\dvar{y} - N\dvar{x} = \int\limits_{a}^{b} \left(M\frac{\dvar{y}}{\dvar{t}}-N\frac{\dvar{x}}{\dvar{t}}\right)\dvar{t}
  %     \]
  %   \end{itemize}
        
  % \item \textbf{Suggested Exercises for 14.2:}
  
  %   \begin{itemize}
  %   \item Work over a curve: 7-22
  %   \item Circulation, flow, and flux: 23-28, 37-40
  %   \end{itemize}
    
  % \end{itemize}
  
  % \newpage
  
  % \centerline{\bf 14.3 Path Independence, Potential Functions, and Conservative Fields}
  
  % \begin{itemize}
    
  % \item Several Equivalencies for Conservative Fields
    
  %   The following are all equivalent for piecewise smooth curves and vector fields with continuous first derivatives:
  %   \begin{itemize}
  %     \item $\vect{F}=\<M,N,P\>$ is a \textbf{conservative field}.
  %     \item $\vect{F}\cdot d\vect{r} = M\,dx+N\,dy+P\,dz$ is \textbf{exact}.
  %     \item $\int\vect{F}\cdot d\vect{r}$ is \textbf{path independent}: the value of $\int_C\vect{F}\cdot d\vect{r}$ only depends on the endpoints of the curve $C$.
  %     \item There exists a \textbf{potential function} $f$ such that $\nabla f = \vect{F}$.
  %     \item (Closed Loop Property of Conservative Fields)\newline $\ds \int_C \vect{F} \cdot d\vect{r} = 0$ for every closed loop $C$ in $D$. 
  %     \item (Fundamental Theorem of Line Integrals)\newline $\ds \int_C \vect{F} \cdot d\vect{r} = f(B)-f(A)$ for every path $C$ connecting $A$ to $B$.
  %     \item (Component Test for Conservative Fields)\newline $\ds \frac{\partial P}{\partial y}=\frac{\partial N}{\partial z},\,\frac{\partial M}{\partial z}=\frac{\partial P}{\partial x},\text{ and }\frac{\partial N}{\partial x}=\frac{\partial N}{\partial y}$. 
  %   \end{itemize}

  % \item \textbf{Suggested Exercises for 14.3:}
  
  %   \begin{itemize}
  %   \item Determining if a field is conservative: 1-6
  %   \item Finding potential functions: 7-12
  %   \item Evaluating integrals of differential forms: 13-22
  %   \end{itemize}
    
  % \end{itemize}
  
  % \newpage
  
  % \centerline{\bf 14.4 Green's Theorem in the Plane}
  
  % \begin{itemize}
    
  % \item Gradient Operator

  % \[\nabla = \<\frac{\partial}{\partial x},\frac{\partial}{\partial y},\frac{\partial}{\partial z}\>\]
    
  % \item Divergence
  
  %   \begin{itemize}
  %   \item The \textbf{divergence} of a planar vector field $\vect{F}=\<M,N\>$ is given by \[ \div \vect{F} = \frac{\partial M}{\partial x}+\frac{\partial N}{\partial y} = \nabla\cdot\vect{F} \]
  %   In physics, divergence is often called the \textbf{flux density}.
  %   \end{itemize}
  
  % \item Spin
  
  %   \begin{itemize}
  %   \item The \textbf{spin} of a planar vector field $\vect{F}=\<M,N\>$ is given by \[ \spin \vect{F} = \frac{\partial N}{\partial x}-\frac{\partial M}{\partial y} \]
  %   In physics, spin is often called the \textbf{circulation density}.
  %   \item Spin is also the \textbf{$\veck$-component of curl}, defined in a later section.
  %   \end{itemize}
    
  % \item Simple Curves
  %   \begin{itemize}
  %   \item A curve which does not cross itself is said to be \textbf{simple}.
  %   \end{itemize}

  % \item Green's Theorem in the Plane
  %   \begin{itemize}
  %   \item Let $C$ be a piecewise smooth, simple closed curve enclosing the region $R$ and oriented counter-clockwise. Let $\vect{F}=\<M,N\>$ be a vector field for which $M,N$ have continuous first partial derivatives in an open region containing $R$. Then:
  %     \[\int\limits_C \vect{F}\cdot\vect{n}\,ds = \iint\limits_R \div \vect{F}\dvar{A}\]
  %     \[\int\limits_C \vect{F}\cdot\vect{T}\dvar{s} = \iint\limits_R \spin \vect{F}\dvar{A}\]
  %   \end{itemize}
    
  % \item \textbf{Suggested Exercises for 14.4:}
  
  %   \begin{itemize}
  %   \item Using Green's Theorem to find circulation and flux: 5-14
  %   \item Using Green's Theorem to evaluate line integrals: 17-20
  %   \end{itemize}
    
  % \end{itemize}
  
  % \newpage
  
  % \centerline{\bf 14.5 Surfaces and Area}
  
  % \begin{itemize}
    
  % \item Parametrization of Surfaces
  
  %   \begin{itemize}
  %   \item Vector functions of two variables \[\vect{r}(u,v)=\<x(u,v),y(u,v),z(u,v)\>\] may be used to parametrize surfaces in $xyz$ space.
  %   \end{itemize}
    
  % \item Smooth Vector Functions
  
  %   \begin{itemize}
  %   \item A surface parametrized by $\vect{r}(u,v)$ is called \textbf{smooth} if 
  %     \[\vect{r}_u = \<\frac{\partial x}{\partial u},\frac{\partial y}{\partial u},\frac{\partial z}{\partial u}\>,\, \vect{r}_v = \<\frac{\partial x}{\partial v},\frac{\partial y}{\partial v},\frac{\partial z}{\partial v}\>\]
  %   are continuous and $\vect{r}_u\times\vect{r}_v\not= \vect{0}$ on the interior of the surface. 
  %   \end{itemize}
    
  % \item Surface Area of a Parametrized Surface
  
  %   \begin{itemize}
  %   \item The area of a smooth surface with parametrizing vector function $\vect{r}(u,v)$ for a region $R$ in the $uv$ plane is given by \[A = \iint\limits_R |\vect{r}_u\times\vect{r}_v|\,dA\]
  %   \end{itemize}
    
  % \item Implicit Surface
  
  %   \begin{itemize}
  %   \item Level surfaces $F(x,y,z)=c$ are sometimes called \textbf{implicit surfaces}.
  %   \item If $\vect{p}$ is a unit vector normal a coordinate plane, then the surface area defined by $F(x,y,z)$ bounded by the cylinder given by a region $R$ in that coordinate plane is \[\iint\limits_R \frac{|\nabla F|}{|\nabla F \cdot \vect{p}|}\,dA\]
  %   \end{itemize}
    
  % \item Surface Area Differential
  
  %   \begin{itemize}
  %   \item The integral $\iint\limits_S\, d\sigma$ is used to represent surface area, and $d\sigma$ is known as the surface area differential.
  %    \[d\sigma = |\vect{r}_u\times\vect{r}_v|\,dA = \frac{|\nabla F|}{|\nabla F \cdot \vect{p}|}\,dA\]
  %   \end{itemize}
        
  % \item \textbf{Suggested Exercises for 14.5:}
  
  %   \begin{itemize}
  %   \item Finding parametrizations of surfaces: 1-16
  %   \item Finding surface area: 17-26
  %   \end{itemize}
    
  % \end{itemize}
  
  % \newpage
  
  % \centerline{\bf 14.6 Surface Integrals and Flux}
  
  % \begin{itemize}
    
  % \item Surface Integrals
  
  %   \begin{itemize}
  %   \item The \textbf{surface integral} of a function $G(x,y,z)$ over a surface $S$ is given by \[
  %     \iint\limits_S G(x,y,z)\,d\sigma = \iint\limits_R G(x(u,v),y(u,v),z(u,v)) |\vect{r}_u\times \vect{r}_v|\,dA
  %   \]
  %   \[
  %     = \iint\limits_R G(x,y,z)\frac{|\nabla F|}{|\nabla F \cdot \vect{p}|}\,dA
  %     \]
  %   \end{itemize}
    
  % \item Orientable Surfaces
  %   \begin{itemize}
  %   \item A surface is said to be \textbf{orientable} if it is ``two-sided'' - there exists a continuous normal unit vector field $\vect{n}$ to the surface.
  %   \end{itemize}
    
  % \item Flux in Three Dimensions
  
  %   \begin{itemize}
  %   \item The flux of a three dimensional vector field $\vect{F}$ across an oriented surface $S$ in the direction of $\vect{n}$ is given by the surface integral \[\iint\limits_S \vect{F}\cdot\vect{n}\,d\sigma\]
  %   \end{itemize}
        
  % \item \textbf{Suggested Exercises for 14.6:}
  
  %   \begin{itemize}
  %   \item Evaluating surface integrals: 1-14
  %   \item Three-dimensional flux: 15-24
  %   \end{itemize}
    
  % \end{itemize}
  
  % \newpage
  
  % \centerline{\bf 14.7 Stokes' Theorem}
  
  % \begin{itemize}
    
  % \item Curl
  %   \begin{itemize}
  %   \item The \textbf{curl} of a vector field $\vect{F}$ is defined as \[\textrm{curl}\,\vect{F} = \nabla \times \vect{F} = \textrm{curl}\,\vect{F} = \<\frac{\partial P}{\partial y}-\frac{\partial N}{\partial z},\frac{\partial M}{\partial z}-\frac{\partial P}{\partial x},\frac{\partial N}{\partial x}-\frac{\partial M}{\partial y}\>\]
  %   \item The counterclockwise spin with respect to a unit vector $\vect{u}$ is given by \[\textrm{spin}_{\vect{u}}\,\vect{F} = \textrm{curl}\,\vect{F} \cdot \vect{u} = \nabla \times \vect{F} \cdot \vect{u}\]
  %   \item In particular, in 2D:
  %     \[
  %       \textrm{spin}\,\vect{F} = \frac{\partial N}{\partial x}-\frac{\partial M}{\partial y} = \textrm{spin}_\veck\,\vect{F}
  %     \]
  %   and in 3D:
  %     \[
  %       \textrm{curl}\,\vect{F} = \<\textrm{spin}_{\veci}\,\vect{F}, \textrm{spin}_{\vecj}\,\vect{F}, \textrm{spin}_{\veck}\,\vect{F}\>
  %     \]
  %   \end{itemize}
  
  % \item Stokes' Theorem
  
  %   \begin{itemize}
  %   \item If a curve $C$ in $\mathbb{R}^3$ is the boundary of a surface $S$, and we want to compute the counter-clockwise circulation with respect to unit normal vectors $\vect{n}$ on the surface, we may use \[\int\limits_C \vect{F}\cdot\vect{T}\,ds = \iint\limits_S \textrm{spin}_{\vect{n}}\,\vect{F}\,d\sigma = \iint\limits_S (\textrm{curl}\,\vect{F} \cdot \vect{n})\,d\sigma = \iint\limits_S \nabla \times \vect{F} \cdot \vect{n}\,d\sigma\] 
  %   \end{itemize}
    
  % \item Identities and Properties
  
  %   \begin{itemize}
  %   \item Due to the Mixed Derivative Theorem, \[\textrm{curl}\, \nabla f = \nabla \times \nabla f = \vect{0}\]
  %   \item If $\nabla \times \vect{F} = \vect{0}$ for every point in a region $D$, then \[ \int\limits_C \vect{F}\cdot d\vect{r} = \iint\limits_S \nabla \times \vect{F} \cdot \vect{n}\,d\sigma = 0\] for every curve $C$ and surface $S$ within $D$.
  %   \end{itemize}
        
  % \item \textbf{Suggested Exercises for 14.7:}
  
  %   \begin{itemize}
  %   \item Using Stokes' Theorem: 1-10
  %   \end{itemize}
    
  % \end{itemize}
  
  % \newpage
  
  % \centerline{\bf 14.8 Divergence Theorem and a Unified Theory}
  
  % \begin{itemize}
    
  % \item Divergence Theorem
  
  %   \begin{itemize}
  %   \item Divergence in $\mathbb{R}^3$ is defined as \[\textrm{div}\,\vect{F} = \frac{\partial M}{\partial x}+\frac{\partial N}{\partial y}+\frac{\partial P}{\partial z} = \nabla \cdot \vect{F} \]
  %   \item The Divergence Theorem lets us measure the flux on a closed surface $S$ by integrating over the divergence within its bounded region $D$: \[\text{Flux} = \iint\limits_S \vect{F}\cdot\vect{n}\,d\sigma = \iiint\limits_D \textrm{div}\,\vect{F}\,dV= \iiint\limits_D \nabla\cdot\vect{F}\,dV\]
  %   \end{itemize}
    
  % % \item The Unified Theory
  
  % %   \begin{itemize}
  % %   \item The unified theory notes that in order to compute circulation and flux over a closed curve or surface, we may consider the spin/curl and divergence over the region bounded by that curve or surface.
  % %   \item Let $C$ be a counter-clockwise closed curve in $\mathbb{R}^2$ bounding the region $R$.
  % %     \[\text{Circulation of } \vect{F} \text{ around } C = \iint\limits_R \textrm{spin}\,\vect{F}\,dA = \iint\limits_R \textrm{curl}\,\vect{F}\cdot \veck\,dA\]
  % %     \[\text{Flux of } \vect{F} \text{ across } C = \iint\limits_R \textrm{div}\,\vect{F}\,dA\]
  % %   \item Let $C$ be a closed curve in $\mathbb{R}^3$ counter-clockwise to $\vect{n}$ bounding the surface $S$.
  % %     \[\text{Circulation of } \vect{F} \text{ around } C = \iint\limits_S \textrm{spin}_{\vect{n}}\,\vect{F}\,d\sigma  = \iint\limits_R \textrm{curl}\,\vect{F}\cdot \vect{n}\,d\sigma\]
  % %   \item Let $S$ be a closed surface in $\mathbb{R}^3$ bounding the solid $D$.
  % %     \[\text{Flux of } \vect{F} \text{ across } S = \iiint\limits_D \textrm{div}\,\vect{F}\,dV\]
  % %   \end{itemize}
        
  % \item \textbf{Suggested Exercises for 14.8:}
  
  %   \begin{itemize}
  %   \item Using the Divergence Theorem: 5-16
  %   \end{itemize}
    
  % \end{itemize}

\end{document}





















