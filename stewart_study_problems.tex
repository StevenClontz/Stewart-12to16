\newcommand{\thetitle}{
  Stewart's Calculus Chapter 12-16 | Study Problems
}

\documentclass[12pt]{article}

\usepackage{fancyhdr}

\usepackage{graphicx}

\pdfpagewidth 8.5in
\pdfpageheight 11in

\setlength\topmargin{0in}
\setlength\headheight{0in}
\setlength\headsep{0.2in}
\setlength\textheight{8in}
\setlength\textwidth{6in}
\setlength\oddsidemargin{0in}
\setlength\evensidemargin{0in}
\setlength\parindent{0in}
\setlength\parskip{0.1in} 

\pagestyle{fancy}
\headheight 35pt

\lhead{}
\chead{\thetitle}
\rhead{Page \thepage}

\lfoot{\footnotesize http://github.com/StevenClontz/Stewart-12to16}
\cfoot{}
\rfoot{\footnotesize Last updated on \today}
 
\usepackage{amssymb}
\usepackage{amsfonts}
\usepackage{amsmath}
\usepackage{mathtools}
\usepackage{amsthm}
\usepackage{wasysym} % for \smiley
      
\newcommand{\ds}{\displaystyle}
\newcommand{\vect}[1]{\mathbf{#1}}
\newcommand{\veci}{\mathbf{i}}
\newcommand{\vecj}{\mathbf{j}}
\newcommand{\veck}{\mathbf{k}}
\newcommand{\dvar}[1]{\,d{#1}}
\renewcommand{\d}[1]{\dvar{#1}}
\renewcommand{\div}{\textrm{div}\,}
\newcommand{\spin}{\textrm{spin}\,}
\newcommand{\curl}{\textrm{curl}\,}
\newcommand{\proj}{\textrm{proj}}
\newcommand{\<}{\left<}
\renewcommand{\>}{\right>}

\newcommand{\hruler}{\hrule\vspace{1em}}

\newcommand{\p}{\partial}
\newcommand{\pd}[2]{\frac{\p #1}{\p #2}}

\newcommand{\Arctan}{\text{Arctan }}
\newcommand{\Arcsin}{\text{Arcsin }}
\newcommand{\Arccos}{\text{Arccos }}
\newcommand{\Arcsec}{\text{Arcsec }}
\newcommand{\Arccsc}{\text{Arccsc }}
\newcommand{\Arccot}{\text{Arccot }}

\begin{document}

\centerline{\bf Chapter 12}

  \begin{enumerate}

    \item Find the cosine of the angle between the vectors $\vect{u}$ and $\vect{v}$. (12.3)

      \begin{enumerate}
        \item $\vect{u}=\<4,-3,0\>$\\ $\vect{v}=\<2,6,-3\>$
        \item $\vect{u}=\<1,2,2\>$\\ $\vect{v}=\<0,0,-3\>$
        \item $\vect{u}=\<1,4,2\>$\\ $\vect{v}=\<4,1,-2\>$
        \item $\vect{u}=\<-4,-4,-6\>$\\ $\vect{v}=\<2,2,3\>$
        \item $\vect{u}=\<0,5,-11\>$\\ $\vect{v}=\<2,0,0\>$
        \item $\vect{u}=\<3,2,1\>$\\ $\vect{v}=\<6,4,2\>$
      \end{enumerate}

    \item Find the projection of $\vect{u}$ onto $\vect{v}$. (12.3)

      \begin{enumerate}
        \item $\vect{u}=\<4,-3,0\>$\\ $\vect{v}=\<2,6,-3\>$
        \item $\vect{u}=\<1,2,2\>$\\ $\vect{v}=\<0,0,-3\>$
        \item $\vect{u}=\<1,4,2\>$\\ $\vect{v}=\<4,1,-2\>$
        \item $\vect{u}=\<-4,-4,-6\>$\\ $\vect{v}=\<2,2,3\>$
        \item $\vect{u}=\<0,5,-11\>$\\ $\vect{v}=\<2,0,0\>$
        \item $\vect{u}=\<3,2,1\>$\\ $\vect{v}=\<6,4,2\>$
      \end{enumerate}

    \newpage

    \item Use the cross product to find a vector normal to both $\vect{u}$ and $\vect{v}$. (12.4)

      \begin{enumerate}
        \item $\vect{u}=\<4,-3,0\>$\\ $\vect{v}=\<2,6,-3\>$
        \item $\vect{u}=\<1,2,2\>$\\ $\vect{v}=\<0,0,-3\>$
        \item $\vect{u}=\<1,4,2\>$\\ $\vect{v}=\<4,1,-2\>$
        \item $\vect{u}=\<-4,-4,-6\>$\\ $\vect{v}=\<2,2,3\>$
        \item $\vect{u}=\<0,5,-11\>$\\ $\vect{v}=\<2,0,0\>$
        \item $\vect{u}=\<3,2,1\>$\\ $\vect{v}=\<6,4,2\>$
      \end{enumerate}

    \item Give a vector equation and parametric equations for the line. (12.5)

      \begin{enumerate}
        \item The line passing through $(1,3,-2)$ and parallel to $\<3,0,1\>$.
        \item The line passing through $(-2,0,4)$ and $(1,3,3)$.
        \item The line parallel to $\vect{r}(t)=\<t,2-t,2+t\>$ and passing through $(2,4,5)$.
        \item The line with equation $x=-3z+1$ in the $xz$ plane.
        \item The line normal to the plane with equation $x+y+2z=4$ and passing through $(1,1,1)$.
      \end{enumerate}

    \item Give the distance from the point to the line: (12.5)

      \begin{enumerate}
        \item $(4,5,3)$ to $\vect{r}(t) = \<1+t,2+2t,2t\>$
        \item $(-1,-2,2)$ to $\vect{r}(t) = \<-1+3t,-4,2+4t\>$
        \item $(3,0)$ to $\vect{r}(t) = \<4-4t,7-3t\>$
        \item $(2,6)$ to $\vect{r}(t) = \<-3+3t,1+t\>$
      \end{enumerate}

    \newpage

    \item Give an equation for the plane. (12.5)

      \begin{enumerate}
        \item The plane passing through $(1,3,-2)$ and normal to $\<3,0,1\>$.
        \item The plane passing through $(1,-2,0)$ and parallel to $2x-y+3z=3$.
        \item The plane passing through $(1,1,1)$ and normal to the line with equation $\vect{r}(t)=\<4-3t,t,2+2t\>$.
        \item The plane passing through $(-2,0,4)$, $(1,3,3)$, and $(0,0,2)$.
      \end{enumerate}

    \item Give the distance from the point to the plane: (12.5)

      \begin{enumerate}
        \item $(5,1,1)$ to $x-2y+2z=2$
        \item $(4,-1,3)$ to $3x+4z=4$
        \item $(0,1,1)$ to $-2x-3y-6z=5$
        \item $(-1,5,2)$ to $x+y+z=3$
      \end{enumerate}

    \item Sketch the curve given by the equation in the appropriate coordinate plane, and then sketch the cylinder in $xyz$ space given by the equation. (12.6)

      \begin{enumerate}
        \item $y=x^2$
        \item $x=z^3$
        \item $y=\sin z$
        \item $z=e^x$
        \item $z=\ln y$
        \item $xy=1$
      \end{enumerate}

    \item Sketch the three coordinate plane cross-sections for the quadric surface given by the equation, sketch the surface itself, and give the name of the surface. (12.6)

      \begin{enumerate}
        \item $x^2-y=-z^2$
        \item $y^2+z^2=4-4x^2$
        \item $z^2-9y^2=x^2$
        \item $y^2-z^2=4-4x^2$
        \item $4x^2-y^2-4z^2=16$
        \item $z=y^2-4x^2$
      \end{enumerate}

\newpage
\centerline{\bf Chapter 13}

    \item Give a parametrization of the curve described as a vector function. (13.1)

      \begin{enumerate}
        \item The parabola $y=x^2$ in the $xy$ plane.
        \item The directed line segment beginning at $(1,2,-3)$ and ending at $(0,3,0)$.
        \item The circle $x^2+y^2=9$.
        \item The ellipse $x^2+9y^2=9$.
      \end{enumerate}

    \item Find the limit of the vector function. (13.1)

      \begin{enumerate}
        \item $\ds\lim_{t\to -1} \<\Arctan t, \frac{e^{1+t}}{1-t}\>$
        \item $\ds\lim_{t\to 2} \<t^2-4,\frac{t^2-4}{t-2}\>$
        \item $\ds\lim_{t\to 0} \left(\frac{\sin3t}{4t}\veci + \frac{1-\cos t}{t}\vecj\right)$
        \item $\ds\lim_{t\to \pi/2} \<\sin t, \cos t, \cot t\>$
        \item $\ds\lim_{t\to 0} \left(\frac{e^t}{t+1}\veci+\frac{e^{t}-1}{t}\vecj+\frac{2^{2t}-1}{t}\veck\right)$
        \item $\ds\lim_{t\to 1} \<\frac{3t^2-3}{t+1}, \frac{\sin(2t-2)}{2t-2},\frac{3t^2-3}{t-1}\>$
      \end{enumerate}

    \item Find the derivative $\frac{d\vect{r}}{dt}=\vect{r}'(t)$ of the vector function. (13.2)

      \begin{enumerate}
        \item $\vect{r}(t) = \<t^2,3+t\>$
        \item $\vect{r}(t) = \<3\sin4t,-3\cos4t\>$
        \item $\vect{r}(t) = \<\frac{1}{t^2},\frac{t}{t^2+1}\>$
        \item $\vect{r}(t) = \<3t^2, 4t^3, 2t+1\>$
        \item $\vect{r}(t) = (\ln 2t)\veci + (e^{2t}-2)\vecj + \frac{1}{e^t}\veck$
        \item $\vect{r}(t) = \<\Arcsin t, \Arccsc t, \Arctan t\>$
      \end{enumerate}

    \newpage

    \item Find the indefinite integral $\int\vect{r}(t)\d{t}$ of the vector function (13.2)

      \begin{enumerate}
        \item $\vect{r}(t) = \<3t^2, 4t^3, 2t+1\>$
        \item $\vect{r}(t) = \<2\sin2t,-2\cos2t\>$
        \item $\vect{r}(t) = \<e^t, 2e^{-t}, e^{3t} \>$
        \item $\vect{r}(t) = \<\frac{1}{t+2}, \frac{1}{(t+2)^2}, \frac{2t}{t^2+2}\>$
        \item $\vect{r}(t) = (e^{2t}e^t + 2t)\veci + \frac{\ln t}{t}\veck$
      \end{enumerate}

    \item Solve the differential vector equation to find $\vect{r}(t)$. (13.2)

      \begin{enumerate}
        \item $\vect{r}'(t) = \<3t^2, 2t^3\>$, $\vect{r}(0) = \<3,4\>$
        \item $\vect{r}'(t) = 3\veci+4\vecj-\veck$, $\vect{r}(1)=\veci-\vecj+2\veck$
        \item $\vect{r}'(t) = \<2e^t, 4, \frac{1}{t}\>$, $\vect{r}(\ln 3)=\<5, 0, \ln(\ln 3)\>$
        \item $\vect{r}'(t) = \<\frac{3}{2}\sqrt{t}, 8t, 3t^2+3\>$, $\vect{r}(1) = \<1, -3, 6\>$
        \item $\vect{r}'(t) = \<\frac{1}{1+t^2},\frac{2t}{1+t^2}\>$, $\vect{r}(0) = \<0, 1\>$
      \end{enumerate}

    \item Find the arclength parameter $s(t)$ where $s(0)=0$ and $\frac{ds}{dt}\geq 0$ for the given curve, and use it to find the arclength of the given portion of the curve. (13.3)

      \begin{enumerate}
        \item $\vect{r}(t) = \<1+2t, 2-t, 3-2t\>$, $1\leq t\leq 3$
        \item $\vect{r}(t) = \< 3\sin t, -4t, 3\cos t \>$, $0\leq t\leq 1$
        \item $\vect{r}(t) = \<3e^t, -4e^t\>$, $0\leq t\leq \ln 2$
        \item $\vect{r}(t) = t^3\veci + t^2\vecj + \veck$, $0\leq t\leq \frac{\sqrt{5}}{3}$
        \item $\vect{r}(t) = \<6t, t^3, 3t^2\>$, $-3\leq t\leq -2$
      \end{enumerate}

    \item Find the unit vectors $\vect{T},\vect{N}$ to the curve in terms of the parameter $t$. (13.3)

      \begin{enumerate}
        \item $\vect{r}(t) = \<3\cos2t,3\sin2t \>$
        \item $\vect{r}(t) = \< 3\sin t, -4t, 3\cos t \>$
        \item $\vect{r}(t) = \< \sqrt{2}\sin t,2\cos t,\sqrt{2}\sin t \>$
        \item $\vect{r}(t) = \<e^t, e^t\sin t, e^t\cos t\>$
      \end{enumerate}

    \newpage

    \item Given the information about $\vect{r}(t)$ at a point, evaluate the binormal vector $\vect{B}$ and curvature $\kappa$ at that same point. (13.3)

      \begin{enumerate}
        \item $\frac{d\vect{r}}{dt}=\<3,0,-4\>$, 
              $\frac{d\vect{T}}{dt}=\<0,10,0\>$ \newline
              $\vect{T}=\<\frac{3}{5},0,-\frac{4}{5}\>$, 
              $\vect{N}=\<0,1,0\>$
        \item $\frac{d\vect{r}}{dt}=\<-3,0,3\sqrt{3}\>$, 
              $\frac{d\vect{T}}{dt}=\<-\sqrt{3},0,-1\>$ \newline
              $\vect{T}=\<-\frac{1}{2},0,\frac{\sqrt{3}}{2}\>$, 
              $\vect{N}=\<-\frac{\sqrt{3}}{2},0,-\frac{1}{2}\>$
        \item $\frac{d\vect{r}}{dt}=\<1,1,1\>$, 
              $\frac{d\vect{T}}{dt}=\<\frac{1}{\sqrt{3}},0,-\frac{1}{\sqrt{3}}\>$ \newline
              $\vect{T}=\<\frac{1}{\sqrt{3}},\frac{1}{\sqrt{3}},\frac{1}{\sqrt{3}}\>$, 
              $\vect{N}=\<\frac{1}{\sqrt{2}},0,-\frac{1}{\sqrt{2}}\>$
        \item $\frac{d\vect{r}}{dt}=\< \frac{3\sqrt{2}}{2}, -4, -\frac{3\sqrt{2}}{2}\>$, 
              $\frac{d\vect{T}}{dt}=\<-\frac{3\sqrt{2}}{10},0,-\frac{3\sqrt{2}}{10}\>$\newline
              $\vect{T}=\< \frac{3\sqrt{2}}{10}, -\frac{4}{5}, -\frac{3\sqrt{2}}{10}\>$, 
              $\vect{N}=\<-\frac{\sqrt{2}}{2},0,-\frac{\sqrt{2}}{2}\>$
      \end{enumerate}

    \item Sketch $\vect{r}(t)$ in the plane and plot the point where $t=a$, and then find and sketch $\vect{v}$, $\vect{a}$ at $t=a$ on the curve. (13.4)

      \begin{enumerate}
        \item $\vect{r}(t) = \<t, t^2\>$, $t=2$
        \item $\vect{r}(t) = \<2\sin t,-2\cos t\>$, $t=\pi/2$
        \item $\vect{r}(t) = \<e^{2t},2t\>$, $t=0$
        \item $\vect{r}(t) = \<\sin(\ln t),\cos(\ln t)\>$, $t=1$
      \end{enumerate}

    \item Assuming ideal projectile motion and $g=10\frac{m}{s^2}$, find the following.

      \begin{enumerate}
        \item Height of a projectile shot from the ground at an angle of $\pi/4$ with initial speed $16\sqrt{2}\frac{m}{s}$ after $2$ seconds.
        \item Flight time of a projectile shot from the ground at an angle of $\pi/6$ with initial speed $100\frac{m}{s}$.
        \item Maximum height of a projectile shot from the ground at an angle of $\pi/3$ with initial speed $50\sqrt{3}\frac{m}{s}$.
        \item Total horizontal distance traveled by a projectile shot from the ground at an angle of $\pi/4$ with initial speed $10\sqrt{2}\frac{m}{s}$.
        \item Initial speed of a projectile shot from the ground at an angle of $\pi/3$ which has traveled $60$ meters horizontally after $4$ seconds.
      \end{enumerate}

    \newpage

    \item Find the tangential and normal components of acceleration for the given position function at the given value of $t$.

      \begin{enumerate}
        \item $\vect{r}(t) = \<3t,t^2\>$, $t=2$
        \item $\vect{r}(t) = \<\sin t, \cos t\>$, $t=\pi/4$
        \item $\vect{r}(t) = \<\frac{1}{3}t^3,2t,t^2\>$, $t=1$
        \item $\vect{r}(t) = \< 3\sin t, -4t, 3\cos t \>$, $t=\pi/2$
      \end{enumerate}

\newpage
\centerline{\bf Chapter 14}

    \item Sketch the domain of the function $f$ in the $xy$ plane, sketch and label the three level curves of $f$ within its domain for each given $k$ value, and then sketch the graph of $f$. (14.1)

      \begin{enumerate}
        \item $f(x,y)=2x-y+1$, $k=-3,0,3$
        \item $f(x,y)=4x^2+y^2$, $k=0,4,16$
        \item $f(x,y)=\sqrt{x^2+9y^2}$, $k=0,3,6$
        \item $f(x,y)=\sqrt{1-x^2-y^2}$, $k=0,\frac{1}{\sqrt{2}},1$
        \item $f(x,y)=\sqrt{1-x^2+y^2}$, $k=0,1,\sqrt{2}$ %todo change 1 to 4 on d and e.
        \item $f(x,y)=\ln(4-x^2-y^2)$, $k=\ln1,\ln2,\ln3$
      \end{enumerate}

    \item Sketch the level surface of $f$ for the given value of $k$. (14.1)

      \begin{enumerate}
        \item $f(x,y,z)=x+y+z$, $k=2$
        \item $f(x,y,z)=x^2+y^2+z^2$, $k=9$
        \item $f(x,y,z)=\sqrt{x^2+4y^2+z^2}$, $k=2$
        \item $f(x,y,z)=z-x^2$, $k=3$
      \end{enumerate}

    \item Prove the limit does not exist by comparing two paths of approach. (14.2)

      \begin{enumerate}
        \item $\ds \lim_{P\to(0,0)} \frac{x^2+y^2}{xy}$
        \item $\ds \lim_{P\to(0,0)} \frac{|xy|}{xy}$
        \item $\ds \lim_{P\to(0,0)} \frac{y^6+x^2}{y^3x+y^6}$
        \item $\ds \lim_{P\to(3,4)} \frac{25-x^2-y^2}{7-x-y}$
      \end{enumerate}

    \newpage

    \item Compute the value of the limit. (14.2)

      \begin{enumerate}
        \item $\ds \lim_{P\to(1,-3)} \frac{6-xy}{3x+y+1}$
        \item $\ds \lim_{P\to(0,0)} \frac{2x^2+4y^2}{\sqrt{x^2+2y^2+1}-1}$
        \item $\ds \lim_{P\to(0,0)} \frac{x\sin 2y - \sin 2y}{y-xy}$
        \item $\ds \lim_{P\to(1,2)} \frac{x^2+2xy+y^2-3x-3y}{x+y}$
        \item $\ds \lim_{P\to(1,2)} \frac{x^2+2xy+y^2-3x-3y}{x+y-3}$
      \end{enumerate}

    \item Find all the first-order and second-order partial derivatives of $f$. (14.3)

      \begin{enumerate}
        \item $f(x,y)=4x^2-5y^3+x-1$
        \item $f(x,y)=3x^2y^2-x^3+y^4-7$
        \item $f(x,y)=\sin(x+3y)$
        \item $f(x,y)=e^{xy^2}$
        \item $f(r,\theta)=r\cos(\theta)$
        \item $f(u,v) = \Arctan(uv)$
      \end{enumerate}

    \item Find the linearization $L(x,y)$ of $f(x,y)$ at $(a,b)$, and use it to approximate the value of $f$ at $(c,d)$. (14.4)

      \begin{enumerate}
        \item $f(x,y)=3x^2-2y^3$, $(a,b)=(1,2)$, $(c,d)=(0.9,2.2)$
        \item $f(x,y)=7y+3xy-1$, $(a,b)=(5,1)$, $(c,d)=(5.1,0.9)$
        \item $f(x,y)=2xy-x^2-y^2$, $(a,b)=(3,-1)$, $(c,d)=(2.9,-1.05)$
        \item $f(x,y)=\sqrt{25-x^2-y^2}$, $(a,b)=(-3,0)$, $(c,d)=(-3.04,0.09)$
      \end{enumerate}

    \newpage

    \item Find the given derivative for the given nested functions at the given point. (14.5)

      \begin{enumerate}
        \item Find $\frac{df}{dt}$ at $t=1$: \newline
          $f(x,y,z)=xyz^2$, $x(t)=2t+1$, $y(t)=t^2+1$, $z(t)=1-t^3$
        \item Find $\pd{g}{u}$ at $(u,v)=(2,0)$: \newline
          $g(x,y)=2x+3x^2y$, $x(u,v)=1-u$, $y(u,v)=1-uv$
        \item Find $\frac{df}{dt}$ at $t=\pi/3$: \newline
          $f(x,y)=4x^2+2y$, $x(t)=\cos t$, $y(t)=2\sin^2 t$
        \item Find $\pd{f}{t}$ at $(t,u)=(0,1)$: \newline
          $f(x,y,z)=ye^x+2z$, $x(t,u)=t^2$, $y(t,u)=t+u$, $z(t,u)=u+1$
        \item Find $\frac{dh}{dt}$ at $t=1$: \newline
          $h(x,y)=x+2y$, $x(u,v)=uv$, $y(u,v)=v^2$, $u(t)=t^2$, $v(t)=t+1$
      \end{enumerate}

    \item Use partial derivatives to find the rate of change $\frac{dy}{dx}$ for the equation at the given point. (14.5)

      \begin{enumerate}
        \item $3x^2+5y=8$ at $(1,1)$
        \item $4x^3y=3xy^3+16$ at $(-1,2)$
        \item $-xy^2+y^3=-5x+5$ at $(-3,2)$
        \item $x^3y^4=x^4y^3$ at $(2,2)$
        \item $e^{xy}=\ln(xy+e)$ at $(1,0)$
        \item $\sin(2x+y)=\cos(2x+y)+1$ at $(\pi/8,\pi/4)$
      \end{enumerate}

    \item Find the derivative of $f$ in the direction of the given vector at the given point. (14.6)

      \begin{enumerate}
        \item $f(x,y)=x+2y$, $\vect{A}=\<-4,3\>$, $P_0=(1,3)$
        \item $f(x,y)=xy^2+3y$, $\vect{A}=\<2,2\>$, $P_0=(2,0)$
        \item $f(x,y)=e^{x+xy}$, $\vect{A}=5\veci-12\vecj$, $P_0=(\ln 2,0)$
        \item $f(x,y,z)=x^2+4y^2+z^2$, $\vect{A}=\<3,-2,-6\>$, $P_0=(1,1,2)$
        \item $f(x,y,z)=xz^3+3yz$, $\vect{A}=\<1,-2,2\>$, $P_0=(-2,0,1)$
        \item $f(x,y,z)=\ln(y^2)+4xz$, $\vect{A}=6\veci-8\veck$, $P_0=(3,1,2)$
      \end{enumerate}

    \newpage

    \item Find and label all the points yielding local maximum values, local minimum values, and saddle points for $f$. (14.7)

      \begin{enumerate}
        \item $f(x,y)=x^2+9y^2+3$
        \item $f(x,y)=x^2-2xy+2y^2+4y-3$
        \item $f(x,y)=x^3+3xy+y^3+2$
        \item $f(x,y)=x^3-3xy^2+6y^2+18x^2+1$ %TODO fix this
        \item $f(x,y)=(x^2+y^2)e^{x+y+2}$
        \item $f(x,y)=x^2y-xy^2+12x-12y$
      \end{enumerate}

    \item Find the absolute maximum and absolute mininum value of $f$ within the closed bounded region $R$. (14.7)

      \begin{enumerate}
        \item $f(x,y)=x^2+y^2$, $R:$ square with vertices $(-1,2)$, $(2,2)$, $(2,5)$, $(-1,5)$
        \item $f(x,y)=x^2+y^2-2x-2y$, $R:$ triangle with vertices $(0,0)$, $(2,4)$, $(2,0)$
        \item $f(x,y)=x^2+2y^2+2xy+4x$, $R = \{(x,y):|x|\leq4,|y|\leq4\}$
        \item $f(x,y)=2xy$, $R = \{(x,y):x^2+4y^2\leq 4\}$
      \end{enumerate}

    \item Use Lagrange Multipliers to find the solution to the word problem. (14.8)

      \begin{enumerate}
        \item Find the maximum volume of a rectangular box without a lid which uses $108$ square units of material.
        \item Find the minimum surface area of a right circular cylinder with volume equal to $54\pi$ cubic units. ($V=\pi r^2h$, $SA=2\pi r(r+h)$)
        \item Find the area of the largest rectangle which has its base on the $x$-axis and fits in the triangle with vertices $(-4,0)$, $(0,8)$, $(4,0)$.
        \item Find the highest and lowest points which lay on the curve of intersection for the cylinder $x^2+y^2=8$ and the plane $2x+2y+z=16$.
      \end{enumerate}

\newpage
\centerline{\bf Chapter 15}

    \item Divide $R$ into $2\times 2$ equal pieces and use the midpoint rule to approximate the double integral. (15.1)

      \begin{enumerate}
        \item $\iint_R 2x+2y+4 \d{A}$, $R = \{(x,y) : 0\leq x\leq 4, 0\leq y\leq 2\}$
        \item $\iint_R 3y^2-4xy\d{A}$, $R = \{(x,y) : -1\leq x\leq 3, -3\leq y\leq 1\}$
        \item $\iint_R 3x^2-2y+4\d{A}$, $R = \{(x,y) : 0\leq x\leq 4, -2\leq y\leq 6\}$
        \item $\iint_R \cos(x+y)\d{A}$, $R = \{ (x,y) : 0\leq x\leq \pi/2, 0\leq y\leq \pi/2\}$
        \item $\iint_R 12x^2y\d{A}$, $R = \{ (x,y) : 0\leq x\leq 3, 1\leq y\leq 2\}$
        \item $\iint_R \frac{y}{1+xy}\d{A}$, $R = \{ (x,y) : 0\leq x\leq 2$, $1\leq y\leq 3\}$
      \end{enumerate}

    \item Evaluate the double integral. (15.2)

      \begin{enumerate}
        \item $\iint_R 2x+2y+4 \d{A}$, $R = \{(x,y) : 0\leq x\leq 4, 0\leq y\leq 2\}$
        \item $\iint_R 3y^2-4xy\d{A}$, $R = \{(x,y) : -1\leq x\leq 3, -3\leq y\leq 1\}$
        \item $\iint_R 3x^2-2y+4\d{A}$, $R = \{(x,y) : 0\leq x\leq 4, -2\leq y\leq 6\}$
        \item $\iint_R \cos(x+y)\d{A}$, $R = \{ (x,y) : 0\leq x\leq \pi/2, 0\leq y\leq \pi/2\}$
        \item $\iint_R 12x^2y\d{A}$, $R = \{ (x,y) : 0\leq x\leq 3, 1\leq y\leq 2\}$
        \item $\iint_R \frac{y}{1+xy}\d{A}$, $R = \{ (x,y) : 0\leq x\leq 2$, $1\leq y\leq 3\}$
      \end{enumerate}

    \item Evaluate the iterated integral or double integral of two variables. (15.3)

      \begin{enumerate}
        \item $\ds\int_0^2\int_0^x 11x^2+3y^2\d{y}\d{x}$
        \item $\ds\int_{-1}^2\int_{-1}^{y^2} 20xy\d{x}\d{y}$
        \item $\ds\int_0^4\int_{\sqrt{y}}^2 6x+30y\d{x}\d{y}$
        \item $\ds\int_{1}^{2}\int_{1/x}^{2/x} xe^x \d{x}\d{y}$
        \item $\iint_R 8xy\d{A}$, $R = \{ (x,y) : 0\leq x\leq y,0\leq y\leq 1\}$
        \item $\iint_R \frac{6}{5}y\d{A}$, $R:$ triangle with vertices $(-2,0)$, $(0,1)$, $(3,0)$
      \end{enumerate}

    \newpage

    \item Evaluate the iterated integral of two variables. (15.3)

      \begin{enumerate}
        \item $\ds\int_0^1\int_x^1 \frac{2}{\sqrt{4+y^2}}\d{y}\d{x}$
        \item $\ds\int_0^2\int_y^2 y(8-x^3)^{1/3}\d{x}\d{y}$
        \item $\ds\int_0^1\int_{\sqrt{y}}^1 3\pi \sin(\pi x^3)\d{x}\d{y}$
        \item $\ds\int_0^1\int_{e^x}^e \frac{y}{\ln y}\d{y}\d{x}$
      \end{enumerate}

    \item Find an expression involving iterated integrals for the given area or average value. (15.3)

      \begin{enumerate}
        \item Area of the rectangle with vertices $(-1,0)$, $(2,0)$, $(2,4)$, $(-1,4)$
        \item Area of the parallelogram with vertices $(-1,2)$, $(3,2)$, $(4,1)$, $(0,1)$
        \item Area of the triangle with vertices $(1,3)$, $(1,1)$, and $(2,2)$
        \item Area between $x=4-y^2$ and $x=y^2-4$
        \item Average value of $f(x,y)=e^{x^2y}$ over the square with vertices $(0,0)$, $(2,0)$, $(2,2)$, $(0,2)$
        \item Average value of $f(x,y)=\sin(\frac{x}{2y})$ over the triangle with vertices $(0,1)$, $(1,1)$, $(0,2)$
      \end{enumerate}

    \item Evaulate the iterated integral of three variables. (15.7)

      \begin{enumerate}
        \item $\ds\int_0^1\int_0^1\int_0^1 8xz-y^2\d{y}\d{x}\d{z}$
        \item $\ds\int_1^2\int_0^x\int_x^{2z} 24y\d{y}\d{z}\d{x}$
        \item $\ds\int_{-1}^1\int_{1+y}^{2+y}\int_0^2 z \d{x}\d{z}\d{y}$
        \item $\ds\int_{-\pi}^0\int_0^{\pi/2}\int_0^x -\sin(z) \d{z}\d{y}\d{x}$
        \item $\ds\int_0^1\int_0^1\int_0^1 \frac{2xy^2}{(1+xyz)^3}\d{z}\d{x}\d{y}$
      \end{enumerate}

    \newpage

    \item Find an expression involving iterated integrals for the volume of the given solid. (15.7)

      \begin{enumerate}
        \item The pyramid with vertices $(0,0,0)$, $(3,0,0)$, $(0,2,0)$, and $(0,0,1)$
        \item The solid in the first octant bounded by the coordinate planes, $z=1-y^2$, and $x=4$
        \item The sphere $x^2+y^2+z^2\leq 4$
        \item The solid bounded by the surfaces $z=4-x^2-y^2$ and $z=4x^2+4y^2-16$
      \end{enumerate}

    \item Find a transformation from either the unit square or triangle in the $uv$ plane into the given region $R$ in the $xy$ plane. (15.10)

      \begin{enumerate}
        \item $R:$ parallelogram bounded by $y=3x+1$, $y=3x-3$, $y=x-2$ $y=x-5$
        \item $R:$ triangle bounded by $y=x$, $y=2x$, $y=6-x$
        \item $R:$ square with vertices $(2,1)$, $(-2,3)$, $(0,7)$, $(4,5)$
        \item $R:$ triangle with vertices $(0,-2)$ $(-1,1)$, $(1,3)$
      \end{enumerate}

    \item Evaulate the double integral of variables $x,y$ using the given transformation from the $uv$ plane. (15.10)

      \begin{enumerate}
        \item $\iint_R 2x-y\d{A}$, $\vect{r}(u,v)=\<u+v,2u-v+3\>$ from unit square into the parallelogram $R$ with vertices $(0,3)$, $(1,5)$, $(2,4)$, $(1,2)$
        \item $\iint_R (x+y)(x-y-2)\d{A}$, $\vect{r}(u,v)=\<4-u-v,v-u+2\>$ from unit triangle into the triangle $R$ with vertices $(4,2)$, $(3,1)$, $(2,2)$
        \item $\iint_R (x+y)e^{x^2-y^2}\d{A}$, $\vect{r}(u,v)=\<u+2v,u-2v\>$ from unit square into the rectangle $R$ bounded by $y=x$, $y=x-4$, $y=-x$, $y=2-x$
        \item $\iint_R e^x\cos(\pi e^x)\d{A}$, $\vect{r}(u,v)=\<\ln (u+v+1),v\>$ from unit triangle into the region $R$ bounded by $y=0$, $y=e^x-2$, $y=\frac{e^x-1}{2}$
      \end{enumerate}

    \newpage

    \item Use polar coordinates to evaluate the double integral or iterated integral. (15.4)

      \begin{enumerate}
        \item $\ds\int_{-1}^1\int_{-\sqrt{1-y^2}}^{\sqrt{1-y^2}} 2y\d{x}\d{y}$
        \item $\ds\int_0^1\int_0^x 3xy\d{y}\d{x}$
        \item $\ds\iint_R e^{x^2+y^2}\d{A}$, $R:$ disk with boundary $x^2+y^2=9$
        \item $\ds\int_{0}^4\int_0^{\sqrt{4x-x^2}} \d{y}\d{x}$
      \end{enumerate}

    \item Use cylindrical coordinates to give an equivalent iterated integral which can be directly evaluated. (15.8)

      \begin{enumerate}
        \item $\ds\int_0^3\int_0^{\sqrt{9-y^2}}\int_0^1 2z\d{z}\d{x}\d{y}$
        \item $\ds\iiint_D \sqrt{x^2+y^2}\d{V}$, $D:$ right circular cylinder bounded by $|z|\leq 2$ and $x^2+y^2=1$
        \item $\ds\int_{-2}^2\int_0^{\sqrt{4-x^2}}\int_{\sqrt{x^2+y^2}}^2 \d{z}\d{y}\d{x}$
        \item The volume of the solid bounded by the $xy$ plane and $z=1-x^2-y^2$
      \end{enumerate}

    \item Use spherical coordinates to give an equivalent iterated integral which can be directly evaluated. (15.9)

      \begin{enumerate}
        \item $\ds\int_{-1}^1\int_{-\sqrt{1-y^2}}^{\sqrt{1-y^2}}\int_0^{\sqrt{1-x^2-y^2}} \d{z}\d{x}\d{y}$
        \item $\ds\iiint_D x\d{V}$, $D:$ hemisphere bounded by $x=\sqrt{4-y^2-z^2}$ and the $yz$ plane
        \item $\ds\int_{-1}^{1}\int_{-\sqrt{1-x^2}}^{\sqrt{1-x^2}}\int_{\sqrt{x^2+y^2}}^{\sqrt{2-x^2-y^2}}3xz\d{z}\d{x}\d{y}$
        \item The volume of the ``ice cream cone'' shaped solid \[D = \{(x,y,z): \sqrt{x^2+y^2} \leq z\leq \sqrt{2x-x^2-y^2}\}\]
      \end{enumerate}

  \newpage
  \centerline{\bf Chapter 16}

    \item Evaluate the line integral with respect to arclength. (16.2)

      \begin{enumerate}
        \item $\int_C 2x+y\d{s}$, $C:$ line segment given by $\vect{r}(t)=\<4t+1,4-3t\>$ for $0\leq t\leq 2$
        \item $\int_C z + 2xy\d{s}$, $C:$ line segment from $(0,-1,3)$ to $(2,2,-3)$
        \item $\int_C xy^3\d{s}$, $C:$ arc on the circle $x^2+y^2=4$ from $(2,0)$ to $(1,\sqrt{3})$
        \item $\int_C 2x\d{s}$, $C:$ parabolic arc on $y=x^2$ from $(0,0)$ to $(1,1)$
      \end{enumerate}

    \item Evaluate the line integral with respect to a variable. (16.2)

      \begin{enumerate}
        \item $\int_C 2x+y\d{x}$, $C:$ line segment given by $\vect{r}(t)=\<4t+1,4-3t\>$ for $0\leq t\leq 2$
        \item $\int_C z + 2xy\d{z}$, $C:$ line segment from $(0,-1,3)$ to $(2,2,-3)$
        \item $\int_C xy^3\d{y}$, $C:$ arc on the circle $x^2+y^2=4$ from $(2,0)$ to $(1,\sqrt{3})$
        \item $\int_C 2x\d{y}$, $C:$ parabolic arc on $y=x^2$ from $(0,0)$ to $(1,1)$
      \end{enumerate}

    \item Compute the work done by the force $\vect{F}$ over the curve $C$. (16.2) % brute force

      \begin{enumerate}
        \item $\vect{F}=\<y,x+y\>$, $C:$ line segment from $(1,3)$ to $(-4,-9)$
        \item $\vect{F}=\<z,xy,z\>$, $C:$ line segment from $(0,-1,3)$ to $(2,2,-3)$
        \item $\vect{F}=\<y^2,x^2\>$, $C:$ one counter-clockwise revolution of the circle $x^2+y^2=9$
        \item $\vect{F}=\<y,2y\>$, $C:$ trigonometric arc on $y=\sin x$ from $(0,0)$ to $(\pi,0)$
      \end{enumerate}

    \item Compute the work done by the force $\vect{F}$ over the curve $C$. (16.3) % fund thm

      \begin{enumerate}
        \item $\vect{F}=\<x,y\>$, $C:$ line segment from $(1,1)$ to $(3,-2)$
        \item $\vect{F}=\<yz,xz,xy\>$, $C:$ line segment from $(0,-3,2)$ to $(4,-1,3)$
        \item $\vect{F}=\<4,z^2,2yz\>$, $C:$ curve given by $\vect{r}(t)=\<2^t,\sin (\pi t),4t^2\>$ for \\ $0\leq t\leq 1$
        \item $\vect{F}=\<2x,1\>$, $C:$ counter-clockwise oriented boundary of the unit square
        \item $\vect{F}=\<12x^2y^2+3y,8x^3y+3x\>$, $C:$ one clockwise revolution of the ellipse $x^2+4y^2=4$
        \item $\vect{F}=\<ye^{xy+z},xe^{xy+z},e^{xy+z}\>$, $C:$ curve given by $\vect{r}(t)=\<\frac{1}{1+t^2},\cos t,e^{1-t^2}\>$ \\ for $-1 \leq t \leq 1$
      \end{enumerate}

    \newpage

    \item Compute the work done by the force $\vect{F}$ over the curve $C$. (16.4) % green's thm

      \begin{enumerate}
        \item $\vect{F}=\<x^2+y,x+y\>$, $C:$ boundary of the unit square oriented counter-clockwise
        \item $\vect{F}=\<x,x^2+xy^3\>$, $C:$ boundary of the rectangle \\$R = \{(x,y) : 1\leq x\leq 2, 1\leq y \leq 3\}$ oriented clockwise
        \item $\vect{F}=\<y,2x\>$, $C:$ boundary of the triangle with vertices $(1,2)$, $(3,-2)$, $(-1,-2)$ oriented counter-clockwise
        \item $\vect{F}=\<x+y,x-y\>$, $C:$ boundary of the upper semicircle $0\leq y\leq \sqrt{4-x^2}$ oriented counter-clockwise
      \end{enumerate}

% % (14.3) Compute work/flow of a conservative field.

% \item Compute the flow of the vector field \[\vec{F}=\left<2xy,x^2-z^2,-2yz\right>\] through the curve $\vec{r}(t)=\left<t2^t,3t^3,\cos(\pi t)\right>$ where $0\leq t\leq 1$. (Hint: Use a potential function.)

% \vspace*{8.5in}

% \liner

% \newpage\up

% % (14.3) Line integral of conservative field over a closed loop.

% \item Show that \[\int\limits_C (ye^{xy}-4yz)\,dx+(xe^{xy}-4xz)\,dy+(-4xy)\,dz = 0\] where $C$ is the pentagon in the $xz$ plane with vertices $(1,0,0)$, $(2,0,1)$, $(2,0,3)$, $(0,0,2)$, and $(0,0,0)$ oriented clockwise with respect to the $y$-axis.

% \vspace*{8in}

% \liner

% \newpage\up

% % (14.4) Green's Theorem for Flux

% \item Express the outward flux of \[\vec{F}=\left<x+y,x^2+y^2\right>\] across the triangle with vertices $(0,0)$, $(1,0)$, and $(1,1)$ as a double iterated integral. \textbf{Do not evaluate the integral.}

% \vspace*{8.5in}

% \liner

% \newpage\up

% % (14.5) Parametrizing a Surface

% \item Use spherical coordinates to give a parametrization corresponding to the portion of the surface \[z^2=x^2+y^2\] between the planes $z=1$ and $z=2$.

% \vspace*{8.5in}

% \liner

% \newpage

% % (14.5) Finding Surface Area

% \item Use the cylindrical coordinate-based parametrization \[\vec{r}(\theta,z)=\left<2\cos\theta,2\sin\theta,z\right>\] to express the area of the surface $x^2+y^2=4$ between the planes $x=0$ and $x=2$ as a double iterated integral. \textbf{Do not evaluate the integral.}

% \vspace*{8in}

% \liner

% \newpage\up

% % (14.6) Surface Integral

% \item Use the spherical coordinate-based parametrization \[\vec{r}(\phi,\theta)=\left<\sin\phi\cos\theta,\sin\phi\sin\theta,\cos\phi\right>\] to express the surface integral $\iint\limits_S 3z^2\,d\sigma$ as a double iterated integral of $\phi,\theta$, where $S$ is the upper half of the unit sphere $z=\sqrt{1-x^2-y^2}$. \textbf{Do not evaluate the integral.}


    \end{enumerate}

\end{document}

