\newcommand{\thetitle}{
  Stewart's Calculus Chapter 12-16 | Study Problems
}

\documentclass[12pt]{article}

\usepackage{fancyhdr}

\usepackage{graphicx}

\pdfpagewidth 8.5in
\pdfpageheight 11in

\setlength\topmargin{0in}
\setlength\headheight{0in}
\setlength\headsep{0.2in}
\setlength\textheight{8in}
\setlength\textwidth{6in}
\setlength\oddsidemargin{0in}
\setlength\evensidemargin{0in}
\setlength\parindent{0in}
\setlength\parskip{0.1in} 

\pagestyle{fancy}
\headheight 35pt

\lhead{}
\chead{\thetitle}
\rhead{Page \thepage}

\lfoot{\footnotesize http://github.com/StevenClontz/Stewart-12to16}
\cfoot{}
\rfoot{\footnotesize Last updated on \today}
 
\usepackage{amssymb}
\usepackage{amsfonts}
\usepackage{amsmath}
\usepackage{mathtools}
\usepackage{amsthm}
\usepackage{wasysym} % for \smiley

      \theoremstyle{plain}
      \newtheorem{theorem}{Theorem}
      \newtheorem{lemma}[theorem]{Lemma}
      \newtheorem{corollary}[theorem]{Corollary}
      \newtheorem{proposition}[theorem]{Proposition}
      \newtheorem{conjecture}[theorem]{Conjecture}
      \newtheorem{question}[theorem]{Question}
      \newtheorem{example}[theorem]{Example}

      \newtheorem*{claim}{Claim}
      
      \theoremstyle{definition}
      \newtheorem{definition}[theorem]{Definition}
      
      \theoremstyle{remark}
      \newtheorem{remark}[theorem]{Remark}
      
\newcommand{\ds}{\displaystyle}
\newcommand{\vect}[1]{\mathbf{#1}}
\newcommand{\veci}{\mathbf{i}}
\newcommand{\vecj}{\mathbf{j}}
\newcommand{\veck}{\mathbf{k}}
\newcommand{\dvar}[1]{\,d{#1}}
\renewcommand{\d}[1]{\dvar{#1}}
\renewcommand{\div}{\textrm{div}\,}
\newcommand{\spin}{\textrm{spin}\,}
\newcommand{\curl}{\textrm{curl}\,}
\newcommand{\proj}{\textrm{proj}}
\newcommand{\<}{\left<}
\renewcommand{\>}{\right>}

\newcommand{\hr}{\hrule\vspace{1em}}

\newcommand{\p}{\partial}
\newcommand{\pd}[2]{\frac{\p #1}{\p #2}}

\newcommand{\Arctan}{\text{Arctan }}
\newcommand{\Arcsin}{\text{Arcsin }}
\newcommand{\Arccos}{\text{Arccos }}
\newcommand{\Arcsec}{\text{Arcsec }}
\newcommand{\Arccsc}{\text{Arccsc }}
\newcommand{\Arccot}{\text{Arccot }}

\begin{document}

\centerline{\bf Chapter 12}

  \begin{enumerate}

    \item Find the cosine of the angle between the vectors $\vect{u}$ and $\vect{v}$. (12.3)

      \begin{enumerate}
        \item $\vect{u}=\<4,-3,0\>$\\ $\vect{v}=\<2,6,-3\>$
        \item $\vect{u}=\<1,2,2\>$\\ $\vect{v}=\<0,0,-3\>$
        \item $\vect{u}=\<1,4,2\>$\\ $\vect{v}=\<4,1,-2\>$
        \item $\vect{u}=\<-4,-4,-6\>$\\ $\vect{v}=\<2,2,3\>$
        \item $\vect{u}=\<0,5,-11\>$\\ $\vect{v}=\<2,0,0\>$
        \item $\vect{u}=\<3,2,1\>$\\ $\vect{v}=\<6,4,2\>$
      \end{enumerate}

    \item Find the projection of $\vect{u}$ onto $\vect{v}$. (12.3)

      \begin{enumerate}
        \item $\vect{u}=\<4,-3,0\>$\\ $\vect{v}=\<2,6,-3\>$
        \item $\vect{u}=\<1,2,2\>$\\ $\vect{v}=\<0,0,-3\>$
        \item $\vect{u}=\<1,4,2\>$\\ $\vect{v}=\<4,1,-2\>$
        \item $\vect{u}=\<-4,-4,-6\>$\\ $\vect{v}=\<2,2,3\>$
        \item $\vect{u}=\<0,5,-11\>$\\ $\vect{v}=\<2,0,0\>$
        \item $\vect{u}=\<3,2,1\>$\\ $\vect{v}=\<6,4,2\>$
      \end{enumerate}

    \newpage

    \item Use the cross product to find a vector normal to both $\vect{u}$ and $\vect{v}$. (12.4)

      \begin{enumerate}
        \item $\vect{u}=\<4,-3,0\>$\\ $\vect{v}=\<2,6,-3\>$
        \item $\vect{u}=\<1,2,2\>$\\ $\vect{v}=\<0,0,-3\>$
        \item $\vect{u}=\<1,4,2\>$\\ $\vect{v}=\<4,1,-2\>$
        \item $\vect{u}=\<-4,-4,-6\>$\\ $\vect{v}=\<2,2,3\>$
        \item $\vect{u}=\<0,5,-11\>$\\ $\vect{v}=\<2,0,0\>$
        \item $\vect{u}=\<3,2,1\>$\\ $\vect{v}=\<6,4,2\>$
      \end{enumerate}

    \item Give a vector equation and parametric equations for the line. (12.5)

      \begin{enumerate}
        \item The line passing through $(1,3,-2)$ and parallel to $\<3,0,1\>$.
        \item The line passing through $(-2,0,4)$ and $(1,3,3)$.
        \item The line parallel to $\vect{r}(t)=\<t,2-t,2+t\>$ and passing through $(2,4,5)$.
        \item The line with equation $x=-3z+1$ in the $xz$ plane.
        \item The line normal to the plane with equation $x+y+2z=4$ and passing through $(1,1,1)$.
      \end{enumerate}

    \item Give the distance from the point to the line: (12.5)

      \begin{enumerate}
        \item $(4,5,3)$ to $\vect{r}(t) = \<1+t,2+2t,2t\>$
        \item $(-1,-2,2)$ to $\vect{r}(t) = \<-1+3t,-4,2+4t\>$
        \item $(3,0)$ to $\vect{r}(t) = \<4-4t,7-3t\>$
        \item $(2,6)$ to $\vect{r}(t) = \<-3+3t,1+t\>$
      \end{enumerate}

    \newpage

    \item Give an equation for the plane. (12.5)

      \begin{enumerate}
        \item The plane passing through $(1,3,-2)$ and normal to $\<3,0,1\>$.
        \item The plane passing through $(1,-2,0)$ and parallel to $2x-y+3z=3$.
        \item The plane passing through $(1,1,1)$ and normal to the line with equation $\vect{r}(t)=\<4-3t,t,2+2t\>$.
        \item The plane passing through $(-2,0,4)$, $(1,3,3)$, and $(0,0,2)$.
      \end{enumerate}

    \item Give the distance from the point to the plane: (12.5)

      \begin{enumerate}
        \item $(5,1,1)$ to $x-2y+2z=2$
        \item $(4,-1,3)$ to $3x+4z=4$
        \item $(0,1,1)$ to $-2x-3y-6z=5$
        \item $(-1,5,2)$ to $x+y+z=3$
      \end{enumerate}

    \item Sketch the curve given by the equation in the appropriate coordinate plane, and then sketch the cylinder in $xyz$ space given by the equation. (12.6)

      \begin{enumerate}
        \item $y=x^2$
        \item $x=z^3$
        \item $y=\sin z$
        \item $z=e^x$
        \item $z=\ln y$
        \item $xy=1$
      \end{enumerate}

    \item Sketch the three coordinate plane cross-sections for the quadric surface given by the equation, sketch the surface itself, and give the name of the surface. (12.6)

      \begin{enumerate}
        \item $x^2-y=-z^2$
        \item $y^2+z^2=4-4x^2$
        \item $z^2-9y^2=x^2$
        \item $y^2-z^2=4-4x^2$
        \item $4x^2-y^2-4z^2=16$
        \item $z=y^2-4x^2$
      \end{enumerate}

\newpage
\centerline{\bf Chapter 13}

    \item Give a parametrization of the curve described as a vector function. (13.1)

      \begin{enumerate}
        \item The parabola $y=x^2$ in the $xy$ plane.
        \item The directed line segment beginning at $(1,2,-3)$ and ending at $(0,3,0)$.
        \item The circle $x^2+y^2=9$.
        \item The ellipse $x^2+9y^2=9$.
      \end{enumerate}

    \item Find the limit of the vector function. (13.1)

      \begin{enumerate}
        \item $\ds\lim_{t\to -1} \<\Arctan t, \frac{e^{1+t}}{1-t}\>$
        \item $\ds\lim_{t\to 2} \<t^2-4,\frac{t^2-4}{t-2}\>$
        \item $\ds\lim_{t\to 0} \left(\frac{\sin3t}{4t}\veci + \frac{1-\cos t}{t}\vecj\right)$
        \item $\ds\lim_{t\to \pi/2} \<\sin t, \cos t, \cot t\>$
        \item $\ds\lim_{t\to 0} \left(\frac{e^t}{t+1}\veci+\frac{e^{t}-1}{t}\vecj+\frac{2^{2t}-1}{t}\veck\right)$
        \item $\ds\lim_{t\to 1} \<\frac{3t^2-3}{t+1}, \frac{\sin(2t-2)}{2t-2},\frac{3t^2-3}{t-1}\>$
      \end{enumerate}

    \item Find the derivative $\frac{d\vect{r}}{dt}=\vect{r}'(t)$ of the vector function. (13.2)

      \begin{enumerate}
        \item $\vect{r}(t) = \<x^2,3+t\>$
        \item $\vect{r}(t) = \<3\sin4t,-3\cos4t\>$
        \item $\vect{r}(t) = \<\frac{1}{t^2},\frac{t}{t^2+1}\>$
        \item $\vect{r}(t) = \<3t^2, 4t^3, 2t+1\>$
        \item $\vect{r}(t) = (\ln 2t)\veci + (e^{2t}-2)\vecj + \frac{1}{e^t}\veck$
        \item $\vect{r}(t) = \<\Arcsin t, \Arccsc t, \Arctan t\>$
      \end{enumerate}

    \newpage

    \item Find the indefinite integral $\int\vect{r}(t)\d{t}$ of the vector function (13.2)

      \begin{enumerate}
        \item $\vect{r}(t) = \<3t^2, 4t^3, 2t+1\>$
        \item $\vect{r}(t) = \<2\sin2t,-2\cos2t\>$
        \item $\vect{r}(t) = \<e^t, 2e^{-t}, e^{3t} \>$
        \item $\vect{r}(t) = \<\frac{1}{t+2}, \frac{1}{(t+2)^2}, \frac{2t}{t^2+2}\>$
        \item $\vect{r}(t) = (e^{2t}e^t + 2t)\veci + \frac{\ln t}{t}\veck$
      \end{enumerate}

    \item Solve the differential vector equation to find $\vect{r}(t)$. (13.2)

      \begin{enumerate}
        \item $\vect{r}'(t) = \<3t^2, 2t^3\>$, $\vect{r}(0) = \<3,4\>$
        \item $\vect{r}'(t) = 3\veci+4\vecj-\veck$, $\vect{r}(1)=\veci-\vecj+2\veck$
        \item $\vect{r}'(t) = \<2e^t, 4, \frac{1}{t}\>$, $\vect{r}(\ln 3)=\<5, 0, \ln(\ln 3)\>$
        \item $\vect{r}'(t) = \<\frac{3}{2}\sqrt{t}, 8t, 3t^2+3\>$, $\vect{r}(1) = \<1, -3, 6\>$
        \item $\vect{r}'(t) = \<\frac{1}{1+t^2},\frac{2t}{1+t^2}\>$, $\vect{r}(0) = \<0, 1\>$
      \end{enumerate}

    \item Find the arclength parameter $s(t)$ where $s(0)=0$ and $\frac{ds}{dt}\geq 0$ and for the given curve, and use it to find the arclength of the given portion of the curve. (13.3)

      \begin{enumerate}
        \item $\vect{r}(t) = \<1+2t, 2-t, 3-2t\>$, $1\leq t\leq 3$
        \item $\vect{r}(t) = \< 3\sin t, -4t, 3\cos t \>$, $0\leq t\leq 1$
        \item $\vect{r}(t) = \<3e^t, -4e^t\>$, $0\leq t\leq \ln 2$
        \item $\vect{r}(t) = t^3\veci + t^2\vecj + \veck$, $0\leq t\leq 1$
        \item $\vect{r}(t) = \<6t, t^3, 3t^2\>$, $-3\leq t\leq -2$
      \end{enumerate}

    \item Find the unit vectors $\vect{T},\vect{N}$ to the curve in terms of the parameter $t$. (13.3)

      \begin{enumerate}
        \item $\vect{r}(t) = \< 3\sin t, -4t, 3\cos t \>$
      \end{enumerate}

    \item Given the information about $\vect{r}(t)$ at a point, evaluate the binormal vector $\vect{B}$ and curvature $\kappa$ at that same point. (13.3)

      \begin{enumerate}
        \item $\frac{d\vect{r}}{dt}=\left< \frac{3\sqrt{2}}{2}, -4, -\frac{3\sqrt{2}}{2}\right>$, $\vect{T}=\left< \frac{3\sqrt{2}}{10}, -\frac{4}{5}, -\frac{3\sqrt{2}}{10}\right>$ \newline $\frac{d\vect{T}}{dt}=\left<-\frac{3\sqrt{2}}{10},0,-\frac{3\sqrt{2}}{10}\right>$, $\vect{N}=\left<-\frac{\sqrt{2}}{2},0,-\frac{\sqrt{2}}{2}\right>$
      \end{enumerate}

    \item Sketch $\vect{r}(t)$ in the plane and plot the point where $t=a$, and then find and sketch $\vect{v}$, $\vect{a}$ at $t=a$ on the curve. (13.4)

      \begin{enumerate}
        \item $\vect{r}(t) = \<2\sin t,-2\cos t\>$, $t=\pi/2$
      \end{enumerate}

    \item Assuming ideal projectile motion and $g=10\frac{m}{s^2}$, find the following.

      \begin{enumerate}
        \item Flight time of a projectile shot from the ground at an angle of $\pi/4$ degrees with initial speed $100\frac{m}{s}$.
        \item Maximum height of a projectile shot from the ground at an angle of $\pi/3$ degrees with initial speed $50\frac{m}{s}$.
      \end{enumerate}

    \item Find the tangential and normal components of acceleration for the given position function at the given value of $t$.

      \begin{enumerate}
        \item $\vect{r}(t) = \< 3\sin t, -4t, 3\cos t \>$, $t=\pi/2$
      \end{enumerate}

% % (11.1) Find the velocity function and speed
% \item Find the velocity and acceleration functions associated with the position function \[\vec{r}(t) = \left<2\sin t,-2\cos t\right>\] which corresponds to the circle \[x^2+y^2=4\]

%   \begin{enumerate}
%     \item Compute correct velocity function (5 points)
%     \item Compute correct acceleration function (5 points)
%   \end{enumerate}

% \vspace*{6in}

% \liner
% \newpage\up

% \item Sketch the curve given by $\vec{r}(t) = \left<2\sin t,-2\cos t\right>$ from \#1 along with its velocity and acceleration vectors at $t=\pi$.

%   \begin{enumerate}
%     \item Sketch correct curve (2 points)
%     \item Mark $t=\pi$ accurately (2 points)
%     \item Sketch $\vec{v}$ correctly (3 points)
%     \item Sketch $\vec{a}$ correctly (3 points)
%   \end{enumerate}

% \vspace*{7.5in}

% \liner
% \newpage\up

% % (11.2) Definite Vector Integral

% \item Evaluate $\ds \int_0^{\pi/4} (\sec^2 t)\vec{i}+(6)\vec{j}+\left(e^t\right)\vec{k} \, dt$.

%   \begin{enumerate}
%     \item (3, 6, or 10 points for 1, 2, or 3 correctly evaulated components)
%   \end{enumerate}

% \vspace*{8in}

% \liner
% \newpage\up

% % (11.2) Initial Value Problem

% \item Find $\vec{r}(t)$ given $\vec{r}'(t) = \left<2t, 2t-t^2, 2e^{2t}\right>$ and $\vec{r}(0)=\left<0,7,-1\right>$.

%   \begin{enumerate}
%     \item (3, 6, or 10 points for 1, 2, or 3 correctly evaulated components)
%   \end{enumerate}

% \vspace*{8.3in}

% \liner
% \newpage\up

% % (11.2) Ideal projectile motion

% \item Assume $g=10$ m/s$^2$. What is the flight time of a projectile launched from the ground with an initial speed of $100$ m/s and launch angle of $\frac{\pi}{6}$?

%   \begin{itemize}
%     \item If using formula:
%       \begin{enumerate}
%         \item Write correct formula (2 points)
%         \item Compute corect answer (8 points)
%       \end{enumerate}
%     \item If using calculus:
%       \begin{enumerate}
%         \item Write correct position function (2 points)
%         \item Attempt to use correct method to find answer (6 points)
%         \item Compute correct answer (2 points)
%       \end{enumerate}
%   \end{itemize}

% \vspace*{6.5in}

% \liner

% \newpage\up

% % (11.3) Give the length of the curve.

% \item Give the length of the arc on the curve $\vec{r}(t) = \left< 3\sin t, -4t, 3\cos t \right>$ between $t=0$ and $t=1$.

%   \begin{enumerate}
%     \item Use arclength formula $\ds \int_a^b |\vec{v}(t)| \dvar{t}$ (2 points)
%     \item Compute $\vec{v}(t)$ correctly (2 points)
%     \item Simplify $|\vec{v}(t)|$ correctly (3 points)
%     \item Compute arclength correctly (3 points)
%   \end{enumerate}

% \vspace*{8in}

% \liner
% \newpage\up

% % (11.4) Circle of Curvature

% \item Write the equation of the circle of curvature to a curve at a point $(4,0)$ with curvature $\frac{1}{10}$ and normal vector $\vec{N} = \left<-\frac{3}{5},\frac{4}{5}\right>$.

%   \begin{enumerate}
%     \item Calculate the radius $a = \frac{1}{\kappa}$ correctly (3 points)
%     \item Calculate the center $\<x_0,y_0\> = \vec{r}(t_0) + a\vec{N}$ correctly (3 points)
%     \item Write a correct equation for the circle (4 points)
%   \end{enumerate}

% \vspace*{7.5in}

% \liner
% \newpage\up

% % % (11.5) Tangential and Normal Components of Acceleration
% % \item Suppose the position of a particle is given by $\vec{r}(t) = \left< 3\sin t, -4t, 3\cos t \right>$. Give the tangential and normal components of acceleration of the particle at $t=0$.

% %   \begin{enumerate}
% %     \item Write a correct formula for $a_T = \frac{d^2s}{dt^2} = \frac{d}{dt}|\vec{v}|$ (2 points)
% %     \item Compute $a_T$ correctly (3 points)
% %     \item Write a correct formula for $a_N = \kappa\left(\frac{ds}{dt}\right)^2 = \kappa|\vec{v}|^2 = \sqrt{|\vec{a}|^2 - a_T^2}$ (2 points)
% %     \item Computer $a_N$ correctly (3 points)
% %   \end{enumerate}

% % \vspace*{7in}

% % \liner
% % \newpage\up

% % (11.3 & 11.4) T, N

% \item Find $\vec{T}$, $\vec{N}$ for $\vec{r}(t) = \left< 3\sin t, -4t, 3\cos t \right>$.

%   \begin{enumerate}
%     \item Compute $\vec{v}$ correctly (2 points)
%     \item Compute $\vec{T}$ correctly (3 points)
%     \item Compute $\frac{d\vec{T}}{dt}$ correctly (2 points)
%     \item Compute $\vec{N}$ correctly (3 points)
%   \end{enumerate}

% \vspace*{8in}

% \liner
% \newpage\up

% % (11.4 & 11.5) \tau, \kappa, B

% \item Given a point $(1,1,0)$ on a curve where $\vec{v}=\left< \frac{3\sqrt{2}}{2}, -4, -\frac{3\sqrt{2}}{2}\right>$, $\frac{d\vec{T}}{dt}=\left<-\frac{3\sqrt{2}}{10},0,-\frac{3\sqrt{2}}{10}\right>$,\newline $\vec{N}=\left<-\frac{\sqrt{2}}{2},0,-\frac{\sqrt{2}}{2}\right>$, $\frac{d\vec{B}}{dt}=\left<-\frac{4\sqrt{2}}{5},0,-\frac{4\sqrt{2}}{5}\right>$, compute $\kappa$, $\tau$, and $\vec{B}$ at that point.

%   \begin{enumerate}
%     \item Write a correct formula for each of $\kappa$, $\tau$, and $\vec{B}$ (1, 2, or 4 points)
%     \item Compute $\kappa$, $\tau$, and $\vec{B}$ correctly (2, 4, or 6 points)
%   \end{enumerate}

% \vspace*{8in}

% \liner

% \newpage\up

% % (11.6) v in terms of u_r, u_\theta
% \item Given the polar parametric equations $r(t)=2-\cos t$ and $\theta(t)=2t$, find $\vec{v}$ in terms of $\vec{u}_r$ and $\vec{u}_\theta$ at $t=0$.

%   \begin{enumerate}
%     \item Write correct formula $\vec{v} = \dot{r}\vec{u}_r + r\dot\theta\vec{u}_\theta$ (2 points)
%     \item Compute $\dot{r}$ correctly (3 points)
%     \item Compute $\dot{\theta}$ correctly (3 points)
%     \item Compute correct expression for $\vec{v}$ (2 points)
%   \end{enumerate}

    \end{enumerate}

\end{document}

